\clearpage 
\thispagestyle{empty}
\begin{center}
	\vspace*{\stretch{1}}
	\LARGE{\textbf{ภาคผนวก ก}}
	\vspace*{\stretch{1}}
\end{center}

	\chapter{บันทึกเวลาการปฏิบัติงาน}

	\begin{tabularx}{\linewidth}{|c|c|c|c|}
		\caption{รายงานบันทึกเวลาปฏิบัติงานประจำเดือน มิถุนายน}\label{timeSheetJun} \\
		\hline
		\multicolumn{1}{|c|}{\textbf{วัน/เดือน/ปี}}	&	\multicolumn{1}{c|}{\textbf{เวลาเข้าทำงาน}} &	\multicolumn{1}{c|}{\textbf{เวลากลับ}} &	\multicolumn{1}{c|}{\textbf{หมายเหตุ}} \\
		\hline
		\endfirsthead
		\caption* {\textbf{ตารางที่ \ref{timeSheetJun} (ต่อ)} รายงานบันทึกเวลาปฏิบัติงานประจำเดือน มิถุนายน} \\
		\hline
		\multicolumn{1}{|c|}{\textbf{วัน/เดือน/ปี}}	&	\multicolumn{1}{c|}{\textbf{เวลาเข้าทำงาน}} &	\multicolumn{1}{c|}{\textbf{เวลากลับ}} &	\multicolumn{1}{c|}{\textbf{หมายเหตุ}} \\
		\hline
		\endhead
		\hline
		\endfoot
		1/มิ.ย./2563 &09.00 & 19.02 & \ \\
		2/มิ.ย./2563 &07.43 & 18.45 & \ \\
		4/มิ.ย./2563 &09.00 & 18.00 & \ \\
		5/มิ.ย./2563 &07.31 & 18.12 & \ \\
		8/มิ.ย./2563 &07.22 & 18.51 & \ \\
		9/มิ.ย./2563 &07.38 & 18.16 & \ \\
		10/มิ.ย./2563 &09.00 & 18.00 & \ \\
		11/มิ.ย./2563 &08.30 & 18.48 & \ \\
		12/มิ.ย./2563 &08.21 & 19.13 & \ \\
		15/มิ.ย./2563 &08.53 & 18.21 & \ \\
		16/มิ.ย./2563 &08.42 & 18.13 & \ \\
		17/มิ.ย./2563 &08.46 & 18.13 & \ \\
		18/มิ.ย./2563 &08.52 & 18.12 & \ \\
		19/มิ.ย./2563 &- & - & ลาป่วย \\
		22/มิ.ย./2563 &08.42 & 18.17 & \ \\
		23/มิ.ย./2563 &08.26 & 18.07 & \ \\
		24/มิ.ย./2563 &08.39 & 18.50 & \ \\
		25/มิ.ย./2563 &08.28 & 18.07 & \ \\
		26/มิ.ย./2563 &08.29 & 18.05 & \ \\
		29/มิ.ย./2563 &08.49 & 18.06 & \ \\
		30/มิ.ย./2563 &07.42 & 18.09 & \ \\
		\hline
	\end{tabularx}

	\begin{tabularx}{\linewidth}{|c|c|c|c|}
		\caption{รายงานบันทึกเวลาปฏิบัติงานประจำเดือน กรกฎาคม}\label{timeSheetJul} \\
		\hline
		\multicolumn{1}{|c|}{\textbf{วัน/เดือน/ปี}}	&	\multicolumn{1}{c|}{\textbf{เวลาเข้าทำงาน}} &	\multicolumn{1}{c|}{\textbf{เวลากลับ}} &	\multicolumn{1}{c|}{\textbf{หมายเหตุ}} \\
		\hline
		\endfirsthead
		\caption* {\textbf{ตารางที่ \ref{timeSheetJul} (ต่อ)} รายงานบันทึกเวลาปฏิบัติงานประจำเดือน กรกฎาคม} \\
		\hline
		\multicolumn{1}{|c|}{\textbf{วัน/เดือน/ปี}}	&	\multicolumn{1}{c|}{\textbf{เวลาเข้าทำงาน}} &	\multicolumn{1}{c|}{\textbf{เวลากลับ}} &	\multicolumn{1}{c|}{\textbf{หมายเหตุ}} \\
		\hline
		\endhead
		\hline
		\endfoot
		1/ก.ค./2563 &07.39 & 18.30 & \ \\
		2/ก.ค./2563 &09.00 & 18.00 & \ \\
		3/ก.ค./2563 &09.00 & 18.00 & ลาป่วย \\
		4/ก.ค./2563 &08.45 & 18.00 & \ \\
		6/ก.ค./2563 &08.01 & 19.08 & \ \\
		7/ก.ค./2563 &07.38 & 18.50 & \ \\
		8/ก.ค./2563 &08.46 & 18.44 & \ \\
		9/ก.ค./2563 &07.33 & 18.55 & \ \\
		10/ก.ค./2563 &08.45 & 18.10 & \ \\
		13/ก.ค./2563 &08.50 & 18.00 & \ \\
		14/ก.ค./2563 &08.45 & 18.00 & \ \\
		15/ก.ค./2563 &08.50 & 18.00& \ \\
		16/ก.ค./2563 &08.55 & 18.00 & \ \\
		17/ก.ค./2563 &08.45 & 18.00 &\ \\
		20/ก.ค./2563 &09.00 & 18.00 &\ \\
		21/ก.ค./2563 &07.38 & 18.41 & \ \\
		22/ก.ค./2563 &09.00 & 20.00 & \ \\
		23/ก.ค./2563 &09.00 & 19.00 & \ \\
		24/ก.ค./2563 &08.32 & 18.07 & \ \\
		29/ก.ค./2563 &07.41 & 18.20 & \ \\
		30/ก.ค./2563 &- & - & ลาธุระส่วนตัว \\
		31/ก.ค./2563 &07.46 & 18.14 & \ \\
		\hline
	\end{tabularx}

	\begin{tabularx}{\linewidth}{|c|c|c|c|}
		\caption{รายงานบันทึกเวลาปฏิบัติงานประจำเดือน สิงหาคม}\label{timeSheetAug} \\
		\hline
		\multicolumn{1}{|c|}{\textbf{วัน/เดือน/ปี}}	&	\multicolumn{1}{c|}{\textbf{เวลาเข้าทำงาน}} &	\multicolumn{1}{c|}{\textbf{เวลากลับ}} &	\multicolumn{1}{c|}{\textbf{หมายเหตุ}} \\
		\hline
		\endfirsthead
		\caption* {\textbf{ตารางที่ \ref{timeSheetAug} (ต่อ)} รายงานบันทึกเวลาปฏิบัติงานประจำเดือน สิงหาคม} \\
		\hline
		\multicolumn{1}{|c|}{\textbf{วัน/เดือน/ปี}}	&	\multicolumn{1}{c|}{\textbf{เวลาเข้าทำงาน}} &	\multicolumn{1}{c|}{\textbf{เวลากลับ}} &	\multicolumn{1}{c|}{\textbf{หมายเหตุ}} \\
		\hline
		\endhead
		\hline
		\endfoot
		3/ส.ค./2563 &- & - & ลากลับสถาบัน \\
		4/ส.ค./2563 &- & - & ลากลับสถาบัน \\
		5/ส.ค./2563 &- & - & ลากลับสถาบัน \\
		6/ส.ค./2563 &- & - & ลากลับสถาบัน \\
		7/ส.ค./2563 &- & - & ลากลับสถาบัน \\
		10/ส.ค./2563 &07.40 & 18.03 & \ \\
		11/ส.ค./2563 &- & - & ลากลับสถาบัน \\
		12/ส.ค./2563 &- & - & ลากลับสถาบัน \\
		13/ส.ค./2563 &- & -& ลากลับสถาบัน \\
		14/ส.ค./2563 &- & - & ลากลับสถาบัน \\
		17/ส.ค./2563 &- & - & ลากลับสถาบัน \\
		18/ส.ค./2563 &09.00 & 18.00 & \ \\
		19/ส.ค./2563 &09.00 & 18.00 &\ \\
		20/ส.ค./2563 &09.00 & 18.00 & \ \\
		21/ส.ค./2563 &09.00 & 18.00 & \ \\
		24/ส.ค./2563 &09.00 & 18.00 & \ \\
		25/ส.ค./2563 &09.00 & 18.00 & \ \\
		26/ส.ค./2563 &09.00 & 18.00 & \ \\
		27/ส.ค./2563 &09.00 & 18.00 & \ \\
		28/ส.ค./2563 &07.14 & 18.05 & \ \\
		31/ส.ค./2563 &07.21 & 18.50 & \ \\
		\hline
	\end{tabularx}

	\begin{tabularx}{\linewidth}{|c|c|c|c|}
		\caption{รายงานบันทึกเวลาปฏิบัติงานประจำเดือน กันยายน}\label{timeSheetAugSep} \\
		\hline
		\multicolumn{1}{|c|}{\textbf{วัน/เดือน/ปี}}	&	\multicolumn{1}{c|}{\textbf{เวลาเข้าทำงาน}} &	\multicolumn{1}{c|}{\textbf{เวลากลับ}} &	\multicolumn{1}{c|}{\textbf{หมายเหตุ}} \\
		\hline
		\endfirsthead
		\caption* {\textbf{ตารางที่ \ref{timeSheetSep} (ต่อ)} รายงานบันทึกเวลาปฏิบัติงานประจำเดือน กันยายน} \\
		\hline
		\multicolumn{1}{|c|}{\textbf{วัน/เดือน/ปี}}	&	\multicolumn{1}{c|}{\textbf{เวลาเข้าทำงาน}} &	\multicolumn{1}{c|}{\textbf{เวลากลับ}} &	\multicolumn{1}{c|}{\textbf{หมายเหตุ}} \\
		\hline
		\endhead
		\hline
		\endfoot
		1/ก.ย./2563 &08.32 & 18.00 & \ \\
		2/ก.ย./2563 &07.28 & 18.00  & \ \\
		3/ก.ย./2563 &09.00 & 18.00 & \ \\
		8/ก.ย./2563 &08.32 & 18.07 & \ \\
		9/ก.ย./2563 &08.14 & 18.00  & \ \\
		10/ก.ย./2563 &07.43 & 18.07 & \ \\
		11/ก.ย./2563 &07.49 & 18.05 & \ \\
		14/ก.ย./2563 &06.47 & 18.00 & \ \\
		15/ก.ย./2563 &06.51 & 18.00 & \ \\
		16/ก.ย./2563 &07.49 & 18.12  &\ \\
		17/ก.ย./2563 &09.00 & 18.00 &\ \\
		18/ก.ย./2563 &07.03 & 18.12 & \ \\
		21/ก.ย./2563 &07.21 & 18.08 &\ \\
		22/ก.ย./2563 &06.57 & 18.10 & \ \\
		23/ก.ย./2563 &08.52 & 18.07 & \ \\
		24/ก.ย./2563 &08.48 & 18.06 & \ \\
		25/ก.ย./2563 &08.52 & 18.14 & \ \\
		28/ก.ย./2563 &07.24 & 18.13 & \ \\
		29/ก.ย./2563 &07.24 & 18.13 & \ \\
		30/ก.ย./2563 &07.05 & 18.08 & \ \\
		\hline
	\end{tabularx}

	\begin{tabularx}{\linewidth}{|c|c|c|c|}
		\caption{รายงานบันทึกเวลาปฏิบัติงานประจำเดือน ตุลาคม}\label{timeSheetAugSepOct} \\
		\hline
		\multicolumn{1}{|c|}{\textbf{วัน/เดือน/ปี}}	&	\multicolumn{1}{c|}{\textbf{เวลาเข้าทำงาน}} &	\multicolumn{1}{c|}{\textbf{เวลากลับ}} &	\multicolumn{1}{c|}{\textbf{หมายเหตุ}} \\
		\hline
		\endfirsthead
		\caption* {\textbf{ตารางที่ \ref{timeSheetSep} (ต่อ)} รายงานบันทึกเวลาปฏิบัติงานประจำเดือน ตุลาคม} \\
		\hline
		\multicolumn{1}{|c|}{\textbf{วัน/เดือน/ปี}}	&	\multicolumn{1}{c|}{\textbf{เวลาเข้าทำงาน}} &	\multicolumn{1}{c|}{\textbf{เวลากลับ}} &	\multicolumn{1}{c|}{\textbf{หมายเหตุ}} \\
		\hline
		\endhead
		\hline
		\endfoot
		1/ต.ค./2563 &07.40 & 18.00 & \ \\
		2/ต.ค./2563 &07.57 & 18.00  & \ \\
		3/ต.ค./2563 &08.40 & 18.18 & \ \\
		5/ต.ค./2563 &07.25 & 18.19 & \ \\
		5/ต.ค./2563 &08.14 & 18.00  & \ \\
		6/ต.ค./2563 &07.25 & 18.19 & \ \\
		7/ต.ค./2563 &07.00 & 18.04 & \ \\
		8/ต.ค./2563 &07.26 & 18.04 & \ \\
		9/ต.ค./2563 &08.47 & 18.05 & \ \\
		12/ต.ค./2563 &07.59 & 18.07  &\ \\
		14/ต.ค./2563 &- & - &ลาธุระส่วนตัว \\
		15/ต.ค./2563 &09.00 & 18.00 & \ \\
		16/ต.ค./2563 &08.56 & 18.27 &\ \\
		19/ต.ค./2563 &09.07 & 19.05 & \ \\
		20/ต.ค./2563 &07.24 & 18.11 & \ \\
		21/ต.ค./2563 &08.42 & 18.08 & \ \\
		22/ต.ค./2563 &08.55 & 18.20 & \ \\
		23/ต.ค./2563 &08.55 & 18.19 & \ \\
		26/ต.ค./2563 &08.52 & 18.24 & \ \\
		27/ต.ค./2563 &08.00 & 18.12 & \ \\
		28/ต.ค./2563 &07.03 & 18.05 & \ \\
		29/ต.ค./2563 &08.58 & 18.07 & \ \\
		30/ต.ค./2563 &09.17 & 18.18 & \ \\
		\hline
	\end{tabularx}

\clearpage 
\thispagestyle{empty}
\begin{center}
	\vspace*{\stretch{1}}
	\LARGE{\textbf{ภาคผนวก ข}}
	\vspace*{\stretch{1}}
\end{center}

\chapter{บันทึกรายงานการปฎิบัติงาน}

% \begin{tabularx}{\linewidth}{|c|c|}
% 	\caption{บันทึกรายงานการปฎิบัติงาน ตุลาคม}\label{timeSheetDetailJuly} \\
% 	\hline
% 	\multicolumn{1}{|c|}{\textbf{วัน/เดือน/ปี}}	&	\multicolumn{1}{c|}{\textbf{เวลาเข้าทำงาน}} &	\multicolumn{1}{c|}{\textbf{เวลากลับ}} &	\multicolumn{1}{c|}{\textbf{หมายเหตุ}} \\
% 	\hline
% 	\endfirsthead
% 	\caption* {\textbf{ตารางที่ \ref{timeSheetDetailJuly} (ต่อ)} รายงานบันทึกเวลาปฏิบัติงานประจำเดือน ตุลาคม} \\
% 	\hline
% 	\multicolumn{1}{|c|}{\textbf{วัน/เดือน/ปี}}	&	\multicolumn{1}{c|}{\textbf{เวลาเข้าทำงาน}} &	\multicolumn{1}{c|}{\textbf{เวลากลับ}} &	\multicolumn{1}{c|}{\textbf{หมายเหตุ}} \\
% 	\hline
% 	\endhead
% 	\hline
% 	\endfoot
% 	1/ต.ค./2563 &07.40 & 18.00 & \ \\
% 	2/ต.ค./2563 &07.57 & 18.00  & \ \\
% 	3/ต.ค./2563 &08.40 & 18.18 & \ \\
% 	5/ต.ค./2563 &07.25 & 18.19 & \ \\
% 	5/ต.ค./2563 &08.14 & 18.00  & \ \\
% 	6/ต.ค./2563 &07.25 & 18.19 & \ \\
% 	7/ต.ค./2563 &07.00 & 18.04 & \ \\
% 	8/ต.ค./2563 &07.26 & 18.04 & \ \\
% 	9/ต.ค./2563 &08.47 & 18.05 & \ \\
% 	12/ต.ค./2563 &07.59 & 18.07  &\ \\
% 	14/ต.ค./2563 &- & - &ลาธุระส่วนตัว \\
% 	15/ต.ค./2563 &09.00 & 18.00 & \ \\
% 	16/ต.ค./2563 &08.56 & 18.27 &\ \\
% 	19/ต.ค./2563 &09.07 & 19.05 & \ \\
% 	20/ต.ค./2563 &07.24 & 18.11 & \ \\
% 	21/ต.ค./2563 &08.42 & 18.08 & \ \\
% 	22/ต.ค./2563 &08.55 & 18.20 & \ \\
% 	23/ต.ค./2563 &08.55 & 18.19 & \ \\
% 	26/ต.ค./2563 &08.52 & 18.24 & \ \\
% 	27/ต.ค./2563 &08.00 & 18.12 & \ \\
% 	28/ต.ค./2563 &07.03 & 18.05 & \ \\
% 	29/ต.ค./2563 &08.58 & 18.07 & \ \\
% 	30/ต.ค./2563 &09.17 & 18.18 & \ \\
% 	\hline
% \end{tabularx}

% % \usepackage{longtable}


% \usepackage{longtable}


\begin{longtable}{|l|l|}
	\caption{บันทึกรายงานปฏิบัติงานประจำเดือน มิถุนายน}\label{timeSheetDetailJuly} \\
	\hline
	\textbf{วันที่} & \textbf{รายละเอียดการทำงาน}            \\                                         
	\hline
	\endfirsthead
	\caption* {\textbf{ตารางที่ \ref{timeSheetDetailJuly} (ต่อ)} บันทึกรายงานปฏิบัติงานประจำเดือน มิถุนายน} \\
	\textbf{วันที่} & \textbf{รายละเอียดการทำงาน}            \\                                         
	\hline
	\endhead
	\hline
	\endfoot
	\hline
	01/มิ.ย./2563   & ปฐมนิเทศ                                                                        \\ 
	\hline
	02/มิ.ย./2563   & ปฐมนิเทศ                                                                        \\ 
	\hline
	04/มิ.ย./2563   & DS : เพิ่มข้อมูลที่หลุดของ vendor ไปยัง QPO                                     \\ 
	\hline
	05/มิ.ย./2563   & DS : เพิ่มข้อมูลที่หลุดของ vendor ไปยัง QPO                                     \\ 
	\hline
	08/มิ.ย./2563   & DS : แก้ไข Fastlane ให้ลง Unitprice                                             \\ 
	\hline
	09/มิ.ย./2563   & DS : แก้ไขการเลื่อนเวลาคิวให้ได้วันว่างจริงๆ                                    \\ 
	\hline
	10/มิ.ย./2563   & DS : แก้ไขการเลื่อนเวลาคิวให้ได้วันว่างจริงๆ                                    \\ 
	\hline
	11/มิ.ย./2563   & DS: กดเปลี่ยนประเภทบริการ ระบบไม่จองคิวใหม่ให้                                  \\ 
	\hline
	12/มิ.ย./2563   & DS : แก้ให้การจองบริการลง disount กับ total amount และ cod ไม่เป็น 0            \\ 
	\hline
	15/มิ.ย./2563   & DS : แก้ให้การจองบริการลง disount กับ total amount และ cod ไม่เป็น 0            \\ 
	\hline
	16/มิ.ย./2563   & SMS : แก้ให้ sms สามารถ เพิ่มกลุ่มได้ทุกคน และเวลาดูชื่อต้อง sender bind        \\ 
	\hline
	17/มิ.ย./2563   & DS : ให้ dscenter สามารถเลือกสาขาที่จะจัดส่งได้                                 \\ 
	\hline
	18/มิ.ย./2563   & DS : แก้ให้วันที่เกิดการ callback และหาค่าใหม่                                  \\ 
	\hline
	19/มิ.ย./2563   & ลาไปมหาวิทยาลัยเพื่อนำเสนอผลงาน                                                 \\ 
	\hline
	22/มิ.ย./2563   & PHY : ให้แสดงวันที่และเวลาก่อน lock และหลัง lock data                           \\ 
	\hline
	23/มิ.ย./2563   & PHY : ให้คำณวน show\_vendor ลงไปใน pdf ด้วย                                     \\ 
	\hline
	24/มิ.ย./2563   & TMS : DEV : เพิ่ม log การทำงาน autoroute                                        \\ 
	\hline
	25/มิ.ย./2563   & STU : เพิ่ม EMPNO\_FULL ใน Add token และให้ get user จาก EMPNO\_FULL แทน        \\ 
	\hline
	26/มิ.ย./2563   & STU : เพิ่ม EMPNO\_FULL ใน Add token และให้ get user จาก EMPNO\_FULL แทน        \\ 
	\hline
	29/มิ.ย./2563   & DEV : สร้างส่วนของ Home Decor และแก้ส่วน Dni ใหม่ใช้ด้วยกันโดยแยก channel H, D  \\ 
	\hline
	30/มิ.ย./2563   & DEV : สร้างส่วนของ Home Decor และแก้ส่วน Dni ใหม่ใช้ด้วยกันโดยแยก channel H, D  \\
	\hline
\end{longtable}

\begin{longtable}{|l|l|}
	\caption{บันทึกรายงานปฏิบัติงานประจำเดือน กรกฎาคม}\label{timeSheetDetailJulyJune} \\
	\hline
	\textbf{วันที่} & \textbf{รายละเอียดการทำงาน}            \\                                         
	\hline
	\endfirsthead
	\caption* {\textbf{ตารางที่ \ref{timeSheetDetailJulyJune} (ต่อ)} บันทึกรายงานปฏิบัติงานประจำเดือน กรกฎาคม} \\
	\hline
	\textbf{วันที่} & \textbf{รายละเอียดการทำงาน}            \\                                         
	\hline
	\endhead
	\hline
	\endfoot
	\hline
	01/ก.ค./2563    & DEV : สร้างส่วนของ Home Decor และแก้ส่วน Dni ใหม่ใช้ด้วยกันโดยแยก channel H, D                                                                           \\ 
	\hline
	02/ก.ค./2563    & DEV : สร้างส่วนของ Home Decor และแก้ส่วน Dni ใหม่ใช้ด้วยกันโดยแยก channel H, D                                                                           \\ 
	\hline
	03/ก.ค./2563    & ลาไปมหาวิทยาลัยเพื่อรับทุน                                                                                                                               \\ 
	\hline
	06/ก.ค./2563    & DEV : สร้างส่วนของ Home Decor และแก้ส่วน Dni ใหม่ใช้ด้วยกันโดยแยก channel H, D                                                                           \\ 
	\hline
	07/ก.ค./2563    & \begin{tabular}[c]{@{}l@{}} DS : group DivMng กับ Shippiont ที่เหมือนกัน \\~และเอาที่ status เป็น Y เมื่อค้นหามาใส่ระบบ fleet ว่าง\end{tabular}          \\ 
	\hline
	08/ก.ค./2563    & \begin{tabular}[c]{@{}l@{}} DS : group DivMng กับ Shippiont ที่เหมือนกัน \\~และเอาที่ status เป็น Y เมื่อค้นหามาใส่ระบบ fleet ว่าง\end{tabular}          \\ 
	\hline
	09/ก.ค./2563    & DS : ให้ตอนกำลัง Confirm Qfail กับ ปิดงาน เช็คว่า Q ถูกคอนเฟิร์มไปก่อนรึยัง                                                                              \\ 
	\hline
	10/ก.ค./2563    & DS : ให้ตอนกำลัง Confirm Qfail กับ ปิดงาน เช็คว่า Q ถูกคอนเฟิร์มไปก่อนรึยัง                                                                              \\ 
	\hline
	13/ก.ค./2563    & DS : เพิ่มอีเมลล์ในการแสดงผลหน้า ประวัติคิว และ หน้ารายละเอียดคิว                                                                                        \\ 
	\hline
	14/ก.ค./2563    & DS : เพิ่มอีเมลล์ในการแสดงผลหน้า ประวัติคิว และ หน้ารายละเอียดคิว                                                                                        \\ 
	\hline
	15/ก.ค./2563    & DS : แก้ fleet ที่หายไปจากหน้าจัดการทีมที่เปลี่ยนคันรถ                                                                                                   \\ 
	\hline
	16/ก.ค./2563    & DS : ให้เข้าไปเช็คว่า art qty กับ qty ตรงกันไหมยืนยันนำจ่ายหรือยัง ใน หน้าปิดงาน                                                                         \\ 
	\hline
	17/ก.ค./2563    & DS : ให้เข้าไปเช็คว่า art qty กับ qty ตรงกันไหมยืนยันนำจ่ายหรือยัง ใน หน้าปิดงาน                                                                         \\ 
	\hline
	20/ก.ค./2563    & DS : ให้เข้าไปเช็คว่า art qty กับ qty ตรงกันไหมยืนยันนำจ่ายหรือยัง ใน หน้าปิดงาน                                                                         \\ 
	\hline
	21/ก.ค./2563    & \begin{tabular}[c]{@{}l@{}} DS : แก้ javascript ให้ complete dialog \\ไม่สามรถกดปิดได้กดได้แค่ตกลง และทำให้ mouse scroll ไม่ได้ หน้าปิดงาน\end{tabular}  \\ 
	\hline
	22/ก.ค./2563    & \begin{tabular}[c]{@{}l@{}} SS : แก้ไขข้อมูล total non vat ใน pdf Department Sales Analysis Report\\~และ แก้ Excel รูปแบบข้อมูลบัตรเครดิต\end{tabular}   \\ 
	\hline
	23/ก.ค./2563    & \begin{tabular}[c]{@{}l@{}} SS : แก้ไขข้อมูล total non vat ใน pdf Department Sales Analysis Report \\~และ แก้ Excel รูปแบบข้อมูลบัตรเครดิต\end{tabular}  \\ 
	\hline
	24/ก.ค./2563    & SS : แก้เลขบัตรเครดิตให้เป็น XXXXXX ตรงกลาง 6 ตัวและนำตัวเลขมาใส่                                                                                        \\ 
	\hline
	29/ก.ค./2563    & SS : แก้เลขบัตรเครดิตให้เป็น XXXXXX ตรงกลาง 6 ตัวและนำตัวเลขมาใส่                                                                                        \\ 
	\hline
	30/ก.ค./2563    & SS : ไล่ดูว่ามันเปลี่ยนไปลง tax\_invoice ตอนไหนในตอนที่กด refund                                                                                         \\ 
	\hline
	31/ก.ค./2563    & SS : ไล่ดูว่ามันเปลี่ยนไปลง tax\_invoice ตอนไหนในตอนที่กด refund                                                                                         \\
	\hline
\end{longtable}

\begin{longtable}{|l|l|}
	\caption{บันทึกรายงานปฏิบัติงานประจำเดือน สิงหาคม}\label{timeSheetDetailJulyJuneAug} \\
	\hline
	\textbf{วันที่} & \textbf{รายละเอียดการทำงาน}            \\                                         
	\hline
	\endfirsthead
	\caption* {\textbf{ตารางที่ \ref{timeSheetDetailJulyJuneAug} (ต่อ)} บันทึกรายงานปฏิบัติงานประจำเดือน สิงหาคม} \\
	\hline
	\textbf{วันที่} & \textbf{รายละเอียดการทำงาน}            \\                                         
	\hline
	\endhead
	\hline
	\endfoot
	\hline
	03/ส.ค./2563    & ลากลับสถาบัน                                                                                                                                              \\ 
	\hline
	04/ส.ค./2563    & ลากลับสถาบัน                                                                                                                                              \\ 
	\hline
	05/ส.ค./2563    & ลากลับสถาบัน                                                                                                                                              \\ 
	\hline
	06/ส.ค./2563    & ลากลับสถาบัน                                                                                                                                              \\ 
	\hline
	07/ส.ค./2563    & ลากลับสถาบัน                                                                                                                                              \\ 
	\hline
	10/ส.ค./2563    & ลากลับสถาบัน                                                                                                                                              \\ 
	\hline
	11/ส.ค./2563    & \begin{tabular}[c]{@{}l@{}} SR : ปรับเงื่อนไขการดึงรายงานภาษีอย่างย่อ กรณีดึงข้อมูลที่ \\Center ปรับให้ดึงได้ 5 ปี\end{tabular}                           \\ 
	\hline
	13/ส.ค./2563    & ลากลับสถาบัน                                                                                                                                              \\ 
	\hline
	14/ส.ค./2563    & ลากลับสถาบัน                                                                                                                                              \\ 
	\hline
	17/ส.ค./2563    & ลากลับสถาบัน                                                                                                                                              \\ 
	\hline
	18/ส.ค./2563    & MEETING : กลับมาวันแรก คุยกับ HR                                                                                                                          \\ 
	\hline
	19/ส.ค./2563    & STUDY : ศึกษาเกี่ยวกับ Automate Test                                                                                                                      \\ 
	\hline
	20/ส.ค./2563    & MEETING : คุยปรึกษากับอาจารย์ถึงขนาดของโปรเจค                                                                                                             \\ 
	\hline
	21/ส.ค./2563    & STUDY : ศึกษาข้อมูลเกี่ยวกับ Device Farm                                                                                                                  \\ 
	\hline
	24/ส.ค./2563    & \begin{tabular}[c]{@{}l@{}} STUDY : ศึกษา Appium ซึ่งเป็น Driver Automate Test Mobile \\ที่ใช้กับ Device Farm\end{tabular}                                \\ 
	\hline
	25/ส.ค./2563    & \begin{tabular}[c]{@{}l@{}} STUDY : เขียน timeline ของการทำงานว่าจะแบ่งการทำงานเป็นอะไรยังไง\\และผลลัพธ์ได้อย่างไร\end{tabular}                           \\ 
	\hline
	26/ส.ค./2563    & \begin{tabular}[c]{@{}l@{}} STUDY : ศึกษา JavaScript(NodeJs) เบื้องต้นเพราะ Appium สามารถใช้ \\Nodejs ในการเขียน Script\end{tabular}                      \\ 
	\hline
	27/ส.ค./2563    & \begin{tabular}[c]{@{}l@{}} STUDY : ศึกษาหาข้อมูลเกี่ยวกับ Driver ที่เป็น library ที่ใช้ในการช่วย \\Test Flutter ใน Appium\end{tabular}                   \\ 
	\hline
	28/ส.ค./2563    & \begin{tabular}[c]{@{}l@{}} STUDY : ลองสร้าง Mobile App ที่เขียนขึ้นมาด้วยภาษา Dart(Flutter)\\~เพื่อที่จะนำไปทดลองใช้กับ Appium\end{tabular}              \\ 
	\hline
	31/ส.ค./2563    & \begin{tabular}[c]{@{}l@{}}STUDY : ทดลองเขียน TestScript ด้วย Library appium-flutter-driver \\แล้วนำไปทดสอบกับ Appium โดยทดสอบกับมือถือจริง\end{tabular}  \\
	\hline
\end{longtable}

\begin{longtable}{|l|l|}
	\caption{บันทึกรายงานการปฎิบัติงาน กันยายน}\label{timeSheetDetailJulyJuneAugSep} \\
	\hline
	\textbf{วันที่} & \textbf{รายละเอียดการทำงาน}            \\                                         
	\hline
	\endfirsthead
	\caption* {\textbf{ตารางที่ \ref{timeSheetDetailJulyJuneAugSep} (ต่อ)} บันทึกรายงานปฏิบัติงานประจำเดือน กันยายน} \\
	\hline
	\textbf{วันที่} & \textbf{รายละเอียดการทำงาน}            \\                                         
	\hline
	\endhead
	\hline
	\endfoot
	\hline
	01/ก.ย./2563    & \begin{tabular}[c]{@{}l@{}} STUDY : ศึกษาและทดลอง Mocha ซึ่งเป็น library ช่วยในการทำ \\TestScript ให้เขียนง่ายขึ้น\end{tabular}                                   \\ 
	\hline
	02/ก.ย./2563    & \begin{tabular}[c]{@{}l@{}} STUDY : ทดลองนำเอา Mocha และ TestScript ไปทดสอบใน \\Android Studio\end{tabular}                                                       \\ 
	\hline
	03/ก.ย./2563    & SIT2: Manual Test Web  Mobile                                                                                                                                     \\ 
	\hline
	08/ก.ย./2563    & SIT2: Manual Test Web  Mobile                                                                                                                                     \\ 
	\hline
	09/ก.ย./2563    & SIT2: Manual Test (Android)                                                                                                                                       \\ 
	\hline
	10/ก.ย./2563    & SIT2: Manual Test (Android)                                                                                                                                       \\ 
	\hline
	11/ก.ย./2563    & SIT2: Manual Test (Android)                                                                                                                                       \\ 
	\hline
	14/ก.ย./2563    & SIT2: Manual Test (Android)                                                                                                                                       \\ 
	\hline
	15/ก.ย./2563    & STUDY : ทดลองใช้ Device Farm กับ Appium                                                                                                                           \\ 
	\hline
	16/ก.ย./2563    & SIT2: Manual Test (Mobile) โทรศัพทร์ส่วนตัว                                                                                                                       \\ 
	\hline
	17/ก.ย./2563    & SIT2: Manual Test (Mobile) โทรศัพทร์ส่วนตัว                                                                                                                       \\ 
	\hline
	18/ก.ย./2563    & STUDY : ทดสอบและติดตั้ง ระบบ ลง window (ที่ผ่านมาใช้บน Ubuntu)                                                                                                    \\ 
	\hline
	21/ก.ย./2563    & STUDY : ศึกษาและทดลองหาวิธีในการทดสอบแอพโดยการไม่แก้ไขโค้ดเดิม (ทำไม่ได้)                                                                                         \\ 
	\hline
	22/ก.ย./2563    & STUDY : ลองแก้ไขโปรเจคและใส่ Key ให้ทำงานกับ Test Script ได้                                                                                                      \\ 
	\hline
	23/ก.ย./2563    & SIT2: เขียน Test Script ทดสอบหมวด Log-in                                                                                                                          \\ 
	\hline
	24/ก.ย./2563    & \begin{tabular}[c]{@{}l@{}} SIT2: Manual Testหาวิธีแก้การต้องกด Permission Allow ทุกครั้งเมื่อรัน App Test \\ให้สามารถ Allow ด้วยตัวเองได้ไม่ต้องกด\end{tabular}  \\ 
	\hline
	25/ก.ย./2563    & SIT2: Manual Testเขียน Test Script หน้าหลักในการ เช็ค Category ของ E-Catalog                                                                                      \\ 
	\hline
	28/ก.ย./2563    & SIT2: Manual Testเขียน Test Script หน้าหลักในการ เช็ค Category ของ E-Catalog                                                                                      \\ 
	\hline
	29/ก.ย./2563    & SIT2: จัดรูปแบบโค้ดใหม่และทำ Search Bar Test                                                                                                                      \\ 
	\hline
	30/ก.ย./2563    & SIT2: Test หมวดย่อย level2 (เครื่องใช้ไฟฟ้า)                                                                                                                      \\
	\hline
\end{longtable}

\begin{longtable}{|l|l|}
	\caption{บันทึกรายงานการปฎิบัติงาน ตุลาคม}\label{timeSheetDetailJulyJuneAugSepOct} \\
	\hline
	\textbf{วันที่} & \textbf{รายละเอียดการทำงาน}            \\                                         
	\hline
	\endfirsthead
	\caption* {\textbf{ตารางที่ \ref{timeSheetDetailJulyJuneAugSepOct} (ต่อ)} บันทึกรายงานปฏิบัติงานประจำเดือน ตุลาคม} \\
	\hline
	\textbf{วันที่} & \textbf{รายละเอียดการทำงาน}            \\                                         
	\hline
	\endhead
	\hline
	\endfoot
	\hline
	01/ต.ค./2563    & SIT2: เช็ค Category Level2 เพิ่มอีก 6 category                                                                                                      \\ 
	\hline
	02/ต.ค./2563    & SIT2: เช็ค Category Level2 ที่เหลือจนเสร็จ                                                                                                          \\ 
	\hline
	05/ต.ค./2563    & SIT2: เช็ค ปุ่มกด"ทั้งหมด"ใน Category ว่าถูกไหม                                                                                                     \\ 
	\hline
	06/ต.ค./2563    & SIT2: เทสการกรองหน้าทั้งหมวด (ยกเว้นกรองแบบ ช่วงราคา)                                                                                               \\ 
	\hline
	07/ต.ค./2563    & SS : แก้ปัญหา SQL ตอนกดดาวน์โหลด execel ใน MYCENTER                                                                                                 \\ 
	\hline
	08/ต.ค./2563    & \begin{tabular}[c]{@{}l@{}} SS : แก้ MY01 ตอนกรอกที่อยู่ให้ที่อยู่เลือกภาษาไทยหรืออังกฤษได้\\และแก้ HTML ที่กรอกที่อยู่\end{tabular}                \\ 
	\hline
	09/ต.ค./2563    & \begin{tabular}[c]{@{}l@{}} SS : แก้ MY01 ตอนกรอกที่อยู่ให้ที่อยู่เลือกภาษาไทยหรืออังกฤษได้\\และแก้ HTML ที่กรอกที่อยู่\end{tabular}                \\ 
	\hline
	12/ต.ค./2563    & SIT2 : ทำ Test Script ในหมดของรายละเอียดสินค้า                                                                                                      \\ 
	\hline
	14/ต.ค./2563    & ลาไปรับรถ                                                                                                                                           \\ 
	\hline
	15/ต.ค./2563    & DS : หาจุดที่ API ผิดพลาดตอนประเมิณช่าง                                                                                                             \\ 
	\hline
	16/ต.ค./2563    & DS : ในตอนจะยืนยันช่างปิดงานด้วยรูปไม่มีให้เลือกว่า ผ่านหรือไม่ผ่าน                                                                                 \\ 
	\hline
	19/ต.ค./2563    & SIT2 : ทำ Test Script ในหมดของรายละเอียดสินค้า                                                                                                      \\ 
	\hline
	20/ต.ค./2563    & SIT2 : ทำ Test Script ในหมดของรายละเอียดสินค้า                                                                                                      \\ 
	\hline
	21/ต.ค./2563    & DS : ในตอนจะยืนยันช่างปิดงานด้วยรูปไม่มีให้เลือกว่า ผ่านหรือไม่ผ่าน                                                                                 \\ 
	\hline
	22/ต.ค./2563    & \begin{tabular}[c]{@{}l@{}} DS : หาว่าเมื่อจัดทีมช่างและนำคิวย้ายช่างส่งเข้าไปถึงไหนในระบบ TMS \\และสถานะที่เปลี่ยนเป็น CC อยู่ที่ไหน\end{tabular}  \\ 
	\hline
	23/ต.ค./2563    & \begin{tabular}[c]{@{}l@{}} DS : หาว่าเมื่อจัดทีมช่างและนำคิวย้ายช่างส่งเข้าไปถึงไหนในระบบ TMS \\และสถานะที่เปลี่ยนเป็น CC อยู่ที่ไหน\end{tabular}  \\ 
	\hline
	26/ต.ค./2563    & SIT2 : ทำ Test Script เปลี่ยนวิธีมาใช้ Xpath ในหมวดของ เปรียบเทียบ                                                                                  \\ 
	\hline
	27/ต.ค./2563    & SIT2 : ทำ Test Script เปลี่ยนวิธีมาใช้ Xpath ในหมวดของ เปรียบเทียบ                                                                                  \\ 
	\hline
	28/ต.ค./2563    & \begin{tabular}[c]{@{}l@{}} DS : แก้ให้ comma สามารถลงคอลัมได้ตอนนำ csv ออกมา และ หาจุด \\error ในหน้า แก้ไขข้อมูลลูกค้า\end{tabular}               \\ 
	\hline
	29/ต.ค./2563    & DEV : เพิ่ม Report Sales แบบ API อยู่ในโปรเจค SalesInformation                                                                                      \\ 
	\hline
	30/ต.ค./2563    & DEV : เพิ่ม Report Sales แบบ API อยู่ในโปรเจค SalesInformation                                                                                      \\
	\hline
\end{longtable}
	
	

\clearpage 
\thispagestyle{empty}
\begin{center}
	\vspace*{\stretch{1}}
	\LARGE{\textbf{ภาคผนวก ค}}
	\vspace*{\stretch{1}}
\end{center}

\chapter{ประวัติผู้เขียน}
\begin{tabularx}{\linewidth}{lX}
	ชื่อ-นามสกุล&นาย เสฎฐวุฒิ ไม้สนธิ์\\
	วัน เดือน ปี เกิด&13 สิงหาคม พ.ศ.2541\\
	ที่อยู่& บ้านเลขที่ 138 ซอยสายลม ถนนพหลโยธิน 
	\\
	&แขวงสามเสนใน เขตพญาไท กรุงเทพมหานคร 10400
	\\
	ประวัติการศึกษา&วิทยาศาสตรบัณฑิต สาขาวิชาเทคโนโลยีสารสนเทศ
	\\
	&คณะเทคโนโลยีสารสนเทศ
	\\
	&สถาบันเทคโนโลยีพระจอมเกล้าเจ้าคุณทหารลาดกระบัง
	\\
	\\
\end{tabularx}