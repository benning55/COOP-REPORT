\documentclass{itkmitlcoop}

\usepackage{afterpage}
\usepackage{graphicx,amsmath,latexsym,amssymb,amsthm}
\usepackage{indentfirst}
\usepackage{cite}
\graphicspath{ {images/} }
\usepackage{tabto}
\usepackage{tabularx}
\usepackage{ltablex}
\usepackage{longtable}
\usepackage[labelfont=bf]{caption}
\usepackage[labelfont=bf,tableposition=top]{caption}
\usepackage{float}
\captionsetup[table]{singlelinecheck=off}
\captionsetup[table]{labelsep=space}
\captionsetup[figure]{labelsep=space}

\makeindex
% ##### Set indent after new line on bibliography
\makeatletter
% \patchcmd{<cmd>}{<search>}{<replace>}{<succes>}{<failure>}
\patchcmd{\@chapter}{\addtocontents{lof}{\protect\addvspace{10\p@}}}{}{}{}% LoF
\patchcmd{\@chapter}{\addtocontents{lot}{\protect\addvspace{10\p@}}}{}{}{}% LoT
\makeatother
%1. Your thesis title (THAI)
\newcommand{\ThesisTiTle}{การพัฒนาชุดคำสั่งทดสอบอัตโนมัติ ด้วย แอพเพี่ยม บนฟลัตเตอร์ กรณีศึกษาของ โฮมโปร อีแคตตาล็อค แอปพลิเคชัน}
%2. Your thesis title (ENG)
\newcommand{\ThesisTiTleEng}{AUTOMATE TESTING WITH APPIUM ON FLUTTER WITH CASE STUDY BY HOMEPRO E-CATALOG APPLICATION}
%3. Your name
\newcommand{\AuName}{เสฎฐวุฒิ ไม้สนธิ์}
\newcommand{\AuNameEng}{Sedthawuth Maisonti}
%4. Your name ENG
\newcommand{\AuNameENG}{SEDTHAWUTH MAISONTI}
\newcommand{\AuNameENGSnakecase}{Sedthawuth Maisonti}
%5. Your student ID
\newcommand{\SId}{60070109}
%6. Your advisor
\newcommand{\Advisor}{ผศ.ดร.บุญประเสริฐ สุรักษ์รัตนสกุล}
\newcommand{\AdvisorEng}{Asst. Prof. Dr. Boonprasert Surakratanasakul}
%7. Your advisor employee
\newcommand{\Exami}{คุณ กรรณิการ์ จันทร์คง}
\newcommand{\ExamiEng}{ test }
%8. สถานประกอบการ
\newcommand{\Company}{บริษัท โฮมโปรดักส์ เซ็นเตอร์ จำกัด (มหาชน)}
%9. ภาคเรียนที่ (in normal letters)
\newcommand{\Sem}{1}
%10. ปีการศึกษา (in normal letters)
\newcommand{\AcaY}{2563}
%11. ปีการศึกษา (in normal letters)
\newcommand{\AcaYAD}{2020}
%12. วันส่งรายงาน
\newcommand{\SubD}{23 พฤศจิกายน พ.ศ. 2563}
%13. วันเริ่มทำงาน
\newcommand{\StartDWork}{1 มิถุนายน พ.ศ. 2563}
%14. วันสุดท้ายของการทำงาน
\newcommand{\EndDWork}{30 พฤศจิกายน พ.ศ. 2563}
%15. ที่อยู่สถานประกอบการ
\newcommand{\Address}{96/27 หมู่ที่ 9 ต.บางเขน อ.เมือง จ.นนทบุรี 11000}
%16. เว็บไซต์สถานประกอบการ
\newcommand{\Website}{www.homepro.co.th}
%17. ตำแหน่งานที่ปฏิบัติ
\newcommand{\Position}{PROGRAMMER}

\newcommand{\Department}{เทคโนโลยีสารสนเทศ}
\newcommand{\DepartmentEng}{Information Technology}

\newcommand{\Branch}{เทคโนโลยีสารสนเทศ}

\begin{document}    
    \pagestyle{empty}
    \frontmatter

    \makecover    
    \makeinnercover
    \makeinnercovereng
    \makecopyright
    \makeletter
    \makeack{
        \begin{enumerate}
            \item คุณ อุบลรัตน์ พรหมสาขา ณ สกลนคร \quad ตำแหน่ง ผู้จัดการทั่วไป-สายบริการส่งเสริมการขาย 
            \item คุณ กรรณิการ์ จันทร์คง \quad ตำแหน่ง ผู้จัดการฝ่ายวิเคราะห์และพัฒนาระบบสนับสนุนการขาย (พนักงานที่ปรึกษา)
        \end{enumerate}
    }
    \makeapproveletter
   
    % Setting margin for page numbering on frontmatter
    \newgeometry{top=1in, bottom=1in, left=1.5in, right=1in, includefoot}

    \makeabstract{
        รายงานเล่มนี้จะนำเสนอผลลัพธ์ และขั้นตอนการดำเนินงานเพื่อให้ได้ชุดคำสั่งทดสอบอัตโนมัติที่ไว้ใช้
        ทดสอบกับแอปพลิเคชัน โฮมโปร อีแคตตาล็อค ที่ใช้กับระบบปฎิบัติการ Android ซึ่งนักศึกษาได้ปฎิบัติสหกิจศึกษา ณ {\Company}
        ตำแหน่ง Programmer {\Company} ได้ทำการพัฒนาแอปพลิเคชันออกมาหลายตัวหนึ่งในนั้นคือ แอปพลิเคชัน โฮมโปร อีแคตตาล็อค ที่ถูกพัฒนาขึ้นมาด้วย
        Flutter เมื่อจะนำขึ้นมาใช้งานจริงจะต้องทำการทดสอบก่อนเสมอจึงทำการทดสอบแบบ Manual ทำให้ใช้เวลานาน
        นักศึกษาจึงทำการศึกษาเครื่อมือที่สามารถนำมาใช้พัฒนาชุดคำสั่งทดสอบอัตโนมัติโดยจะเป็นการใช้ APPIUM เป็นตัวขับเคลื่อนการทำงานของชุดคำสั่งทดสอบอัตโนมัติ
        ที่ถูกสร้างขึ้นด้วย NodeJs เป็นภาษา JavaScript และทดลองนำชุดคำสั่งทดสอบอัตโนมัติไปทดสอบบน AWS Device Farm เพื่อจำลองการทดสอบแอปพลิเคชัน
        ในอุปกรณ์ที่หลากหลาย
    }
    \pagestyle{fancy}
	\pagenumbering{Roman}
   \makeabstracteng{
       This report will offer result and development process to build automate test script for application HomePro E-Catalog. That run on android OS
       which cooperative education student work in Home Product Center Public Company Limited as Programmer.
       Home Product Center Public Company Limited create a lot of application which one of that is HomePro E-Catalog that are made with Flutter Framework.
       Before application go to production. The application need to be tested manual which take a longtime, so cooperative education student
       try to discover tool that will provide to run automate test wich found to be APPIUM that run test script that wrote with NodeJs as JavaScript language, then move
       test script to test on  AWS Device Farm for variety Device to be tested.
   }

    \newpage
    \addcontentsline{toc}{chapter}{สารบัญ}
    \tableofcontents
    
    \newpage
    \addcontentsline{toc}{chapter}{สารบัญตาราง}
    \listoftables    
    
    \newpage
    \addcontentsline{toc}{chapter}{สารบัญรูป}
    \listoffigures
    
    % Reset frontmatter page numbering margin, back to original margin from class file
    \restoregeometry

    \mainmatter
    \pagestyle{plain}
    \pagenumbering{arabic}
    \chapter{บทนำ}
\thispagestyle{empty}
\label{chapter:introduction}

\section{ที่มาและความสำคัญ}
    {\Company} เป็นบริษัทขายปลีกที่จำหน่ายสินค้าและให้บริการที่เกี่ยวข้องกับการก่อสร้าง ตกแต่ง ต่อเติม ซ่อมแซม ปรับปรุง อาคาร บ้าน และที่อยู่อาศัยแบบครบวงจร
    แต่หากว่า โฮมโปรหน้าร้านบางสาขาอาจจะไม่มีสินค้าที่ลูกค้าต้องการอยู่เพราะขนาดของสาขาของโฮมโปรมีหลายขนาดตั้งแต่ร้านขนาดเล็กที่ตั้งอยู่ในห้างสรรพสินค้าของเจ้าอื่น
    ไปจนถึงตั้งแยกออกมาเป็นห้างสรรพสินค้าขนาดใหญ่ของตนเองจึงทำให้สินค้าที่มีอยู่ไม่สามารถแสดงได้ทั้งหมดตามขนาดโฮมโปรจึงจัดทำแอปพลิเคชัน E-Catalog ขึ้นมา
    แอปพลิเคชัน E-Catalog คือแอปพลิเคชันในการสนับสนุนการขายของพนักงานสาขาในการที่จะแสดงสินค้าที่โฮมโปรมีแต่สาขาที่ลูกค้าอยู่ไม่มีได้ยกตัวอย่างเช่น โฮมโปร
    สาขา ประชาชื่น ไม่มีแอร์รุ่นที่ลูกค้าต้องการพนักงานขายของสาขาสามารถแสดงรูปตัวอย่างสินค้าและคุณสมบัติของสินค้าผ่านแอปพลิเคชันให้ลูกค้าดูว่าตรงกับความต้องการลูกค้าหรือไม่และแสดง
    ถึงสาขาที่มี แอร์อยู่ได้อีกทั้งยังสามารถสั่งจองสินค้านั้นจากสาขาที่มีผ่านทางแอปพลิเคชันได้เลยเพียงแค่กรอกเบอร์โทรศัพท์ และ แอปพลิเคชันสามารถใช้ในการช่วยขายสินค้าที่เกี่ยวข้องกัน
    ให้กับลูกค้าเพิ่มเติมได้ เช่น ลูกค้าเลือกซื้อประตูพนักงานสาขาสามารถแสดงลูกบิดที่ดูเหมาะสมเข้ากับประตูนั้นได้ เป็นต้น
    เพื่อป้องกันข้อผิดพลาดก่อนนำแอปพลิเคชันไปใช้จริงต้องทดสอบเพื่อสังเกตหาข้อผิดพลาดทุกครั้งโดยแต่ละครั้งหากตัวแอปพลิเคชันเกิดการแก้ไขแม้ว่าจะมากหรือน้อยก็ตาม
    เพราะการแก้ไขแต่ละครั้งอาจจะส่งผลกระทบกับส่วนอื่นๆของแอปพลิเคชันได้ แต่จะเสียแรงงานและเวลากับการทดสอบได้หากเกิดการแก้ไขบ่อยครั้ง
    ดังนั้นบริษัท โฮมโปรดักส์ เซ็นเตอร์ จํากัด (มหาชน) จึงได้เล็งเห็นถึงความสําคัญของการทดสอบ
    แอปพลิเคชันแบบอัตโนมัติ จึงให้นักศึกษาปฏิบัติงานโครงการทวิภาคีสถาบันเทคโนโลยีพระจอมเกล้าเจ้าคุณทหารลาดกระบัง นาย เสฎฐวุฒิ ไม้สนธิ์ ทำการทดสอบโดยการเขียนชุดคำสั่งทดสอบอัตโนมัติ (automate testing) ให้ใช้คู่กับ AWS Device Farm
    ในการพัฒนาชุดคำสั่งทดสอบอัตโนมัติเพื่อลดระยะเวลาและแรงงานในการทดสอบแอปพลิเคชันแบบเดิม

\section{วัตถุประสงค์}
    \begin{enumerate}
        \item เพื่อพัฒนาชุดคำสั่งทดสอบอัตโนมัติ
        \item เพื่อลดข้อผิดพลาดจากการทดสอบแอปพลิเคชันแบบ Manual
        \item เพื่อลดเวลาการทดสอบและส่งมอบแอปพลิเคชัน
        \item เพื่อเป็นต้นแบบในการทดสอบแอปพลิเคชันอื่นของบริษัท
    \end{enumerate}

\section{ประวัติ และรายละเอียดสถานประกอบการ}
    \subsection{ชื่อ และสถานที่ตั้งของสถานประกอบการ}
        {\Company}

        ที่อยู่ {\Company} สำนักงานใหญ่

        {\Address}
    \subsection{ประวัติความเป็นมาของสถานประกอบการ}
        บมจ. โฮม โปรดักส์ เซ็นเตอร์ จดทะเบียนจัดตั้งขึ้นเมื่อวันที่ 27 มิถุนายน 2538 โดยเป็นการร่วมลงทุนของ บมจ. แลนด์แอนด์เฮ้าส์ และ บมจ. ควอลิตี้เฮ้าส์ บริษัทฯ เริ่มต้นเปิดดำเนินการที่สาขารังสิตในเดือนกันยายน 2539 เป็นแห่งแรก โดยใช้ชื่อทางการค้าว่า “โฮมโปร” (HomePro)
        บริษัทฯ ได้จดทะเบียนแปรสภาพเป็นบริษัทมหาชนในวันที่ 29 พฤษภาคม 2544 ด้วยทุนจดทะเบียนเริ่มต้น 150 ล้านบาท ต่อมาได้จดทะเบียนเป็นบริษัทรับอนุญาตในตลาดหลักทรัพย์แห่งประเทศไทยในวันที่ 30 ตุลาคม 2544 โดยใช้ชื่อย่อหลักทรัพย์ว่า “HMPRO”
        ในวันที่ 26 พฤษภาคม 2548 บริษัทฯ ได้จดทะเบียนจัดตั้งบริษัท มาร์เก็ต วิลเลจ จำกัด โดยมีวัตถุประสงค์เพื่อบริหารพื้นที่ให้เช่า พร้อมกับให้บริการทางด้านสาธารณูปโภค เริ่มต้นดำเนินการในไตรมาสแรก ปี 2549 ที่โครงการ “หัวหิน มาร์เก็ต วิลเลจ” (Hua-Hin Market Village)
        และในปี 2549 บริษัทฯ ได้ถูกคัดเลือกให้เป็นหลักทรัพย์ในกลุ่ม SET100
        ในปี 2553 บริษัทฯ ได้รับคัดเลือกให้เป็นหลักทรัพย์ในกลุ่ม SET 50  และได้เปิดดำเนินการครบ 15 ปี มีสาขาทั้งสิ้น 40 แห่ง เป็นสาขาในเขตกรุงเทพฯ และปริมณฑล 19 แห่ง ในต่างจังหวัด 21 แห่ง
        \begin{figure}[H]
            \centering
            \includegraphics[width=0.5\textwidth]{homepro-logo}
            \caption{ตราตราสัญลักษณ์ บริษัท โฮมโปรดักส์ เซ็นเตอร์ จำกัด (มหาชน)}\label{homepro-logo}
        \end{figure}

    \subsection{ลักษณะการประกอบการ ผลิตภัณฑ์/ผลิตผล}
    \begin{enumerate}
        \item ธุรกิจค้าปลีก
        \begin{itemize}
            \item[-] สินค้าที่เกี่ยวกับวัสดุก่อสร้าง สี อุปกรณ์ ปรับปรุงบ้าน ห้องน้ำและสุขภัณฑ์ เครื่องครัว อุปกรณ์ และ เครื่องใช้ไฟฟ้า
            \item[-] สินค้าประเภทเครื่องนอน พรม ผ้าม่าน เฟอร์นิเจอร์ โคมไฟ สินค้าตกแต่ง และอุปกรณ์เครื่องใช้ ภายในบ้าน  
        \end{itemize}
        \item  บริการที่เกี่ยวเนื่องกับธุรกิจค้าปลีก เนื่องจากสินค้าส่วนใหญ่ของบริษัทฯ เป็นสินค้าที่มี รายละเอียดของวิธีการ และขั้นตอนการใช้งานที่ต้องมีการ ถ่ายทอดให้กับลูกค้า บริษัทฯ จึงจัดให้มีบริการด้านต่างๆ ที่เกี่ยวข้อง โดยเริ่มตั้งแต่การให้คำ ปรึกษา และข้อมูลที่จะ เป็นประโยชน์ต่อการตัดสินใจเพื่อให้ลูกค้าสามารถเลือก ซื้อสินค้าได้ตรงกับวัตถุประสงค์การใช้งานมากที่สุด อีกทั้ง ยังมีบริการ “โฮม เซอร์วิส” (Home Service) ที่ให้บริการ ครอบคลุมงานออกแบบห้องด้วยระบบคอมพิวเตอร์ 3 มิติ (3D Design) และงานบริการดังต่อไปนี้
        \begin{itemize}
            \item[-] งานติดตั้ง ย้ายจุด แก้ปัญหา (Installation Service)
            \item[-] งานตรวจเช็ค ทำความสะอาด/บำรุงรักษาเครื่องใช้ ไฟฟ้าต่างๆ (Maintenance Service)
            \item[-] งานปรับปรุง เปลี่ยนแปลงห้องน้ำ ห้องครัว ห้องนั่งเล่น (Home Improvement Service)
            \item[-] งานบริการล้างและทำความสะอาด (Cleaning Service)
            \item[-] งานปรับปรุงบ้าน ปรับปรุงพื้นที่ใช้สอยภายในบ้าน (Home Makeover)    
        \end{itemize}
        \item บริษัทฯ มีการจัดสรรพื้นที่ในบางสาขาเพื่อให้บริการ แก่ร้านค้าเช่า และมีการพัฒนารูปแบบสาขาที่เรียกว่า “มาร์เก็ต วิลเลจ” (Market Village) ซึ่งดำเนินธุรกิจ ในลักษณะของศูนย์การค้าเต็มรูปแบบภายในโครงการ นอกจากจะมีสาขาของโฮมโปรแล้ว ยังมีพื้นที่ในส่วนของ ศูนย์การค้า โดยผู้เช่าส่วนใหญ่ ได้แก่ ซุปเปอร์มาร์เก็ต ร้านอาหาร ธนาคาร ร้านหนังสือ ร้านสินค้าไอที เป็นต้น ณ วันที่ 31 ธันวาคม 2562 บริษัทฯ มีสาขาในรูปแบบ “มาร์เก็ต วิลเลจ” ทั้งสิ้น 4 แห่ง ได้แก่ สุวรรณภูมิ หัวหิน ภูเก็ต (ฉลอง) และราชพฤกษ์
    \end{enumerate}

    \subsection{แบบการจัดการองค์กร และการบริหารงาน}
        \begin{figure}[H]
            \centering
            \includegraphics[width=1\textwidth]{homepro-structure}
            \caption{โครงสร้างองค์กรของ บริษัท โฮมโปรดักส์ เซ็นเตอร์ จำกัด (มหาชน)}\label{homepro-structure}
        \end{figure}

    \subsection{ตำแหน่ง และหน้าที่ของงานที่นักศึกษาได้รับมอบหมาย}
        นักศึกษาได้ทำสหกิจในตำแหน่ง PROGRAMMER มีหน้าพัฒนาชุดพัฒนาคำสั่งทดสอบแอปพลิเคชัน E-Catalog ตามเหตุการณ์ที่ผู้ใช้แอปพลิเคชันต้องเจอ และ 
        นำไปทดสอบบน AWS Device Farm อีกทั้งเป็นผู้ร่วมจัดทำคู่มือ การติดตั้งเครื่องมือในการทำพัฒนาชุดคำสั่งทดสอบอัตโนมัติ
        วิธีการสร้างพัฒนาชุดคำสั่งทดสอบอัตโนมัติ

\section{ชื่อ และตำแหน่งของพนักงานที่ปรึกษา}
    \begin{tabular}{ll}
        ชื่อ&{\Exami}\\
        ตำแหน่ง&ผู้จัดการทั่วไป\\
        แผนก&สายบริการส่งเสริมการขาย
    \end{tabular}

\section{ระยะเวลาที่ปฏิบัติงาน}
    เริ่มปฏิบัติงานสหกิจศึกษาตั้งแต่วันที่ 1 มิถุนายน พ.ศ. 2563 ถึง 30 พฤศจิกายน พ.ศ. 2563 รวมเป็นระยะเวลา 26 สัปดาห์

    \chapter{รายละเอียดการปฏิบัติงาน}
\thispagestyle{empty}
\label{chapter:coop-detail}

\section{ตำแหน่ง/หน้าที่ของงานที่ได้รับมอบหมาย}

    \subsection{ตำแหน่งงาน}
        PROGRAMMER
    
    \subsection{หน้าที่ของงานที่ได้รับมอบหมาย}
        \begin{enumerate}
            \item ศึกษาวิธีการทดสอบกับ Flutter แอปพลิเคชันโดยการเพิ่ม key
            \item ศึกษาวิธีการใช้งาน Appium ซึ่งเป็น Framework ที่ช่วยในการทดสอบ แอปพลิเคชัน อัตโนมัติ
            \item ศึกษาวิธีการใช้งาน Appium กับ Flutter ผ่าน Flutter Appium Driver Library
            \item ศึกษาการใช้งาน AWS Device Farm ซึ่งเป็นตัวช่วยจำลองโทรศัพท์ในการทดสอบแอปพลิเคชัน
            \item ศึกษาการพัฒนาชุดคำสั่งทดสอบด้วย NodeJs
            \item ศึกษา HomePro E-Catalog แอปพลิเคชนซึ่งเป็นแอปพลิเคชันที่ต้องนำมาทดสอบ
            \item ออกแบบและพัฒนาชุดคำสั่งทดสอบ
            \item จัดทำคู่มือวิธีการติดตั้ง Appium, AWS Device Farm, NodeJs
            \item จัดทำคู่มือวิธีการพัฒนาชุดคำสั่งทดสอบอัตโนมัติด้วย NodeJs กับ Flutter Appium Driver Library
            \item จัดทำคู่มือวิธีการใช้งาน Appium, AWS Device Farm, NodeJs
        \end{enumerate}

\section{รายละเอียดของโครงงานที่ได้รับผิดชอบ}

    เนื่องจากใน {\Company} เป็นบริษัทขนาดใหญ่จึงจำเป็นต้องมีแอปพลิเคชันหรือระบบภายในไว้ใช้งานจึงมีแผนก ICT Non SAP Front Office
    ไว้คอยพัฒนาระบบหรือแอปพลิเคชันโดยในการจะนำเอาแอปพลิเคชันมาใช้งานหรือแก้ไขต้องเกิดการทดสอบก่อนเสมอเพื่อลดข้อผิดพลาดทางแผนกจึงมอบหมายงาน
    ให้พัฒนาการทดสอบแอปพลิเคชันอัตโนมัติ (Automate Testing) ของแอปพลิเคชัน HomePro E-Catalog ที่เป็นแอปพลิเคชันที่สร้างขึ้นด้วย Flutter เพื่อเป็นต้นแบบไว้คอยนำมาประยุกต์ใช้งานกับ
    แอปพลิเคชันที่จะเกิดขึ้นในอนาคต โดยงานหลักแบ่งได้ 2 อย่างได้แก่
    
    \begin{enumerate}
        \item พัฒนาชุดคำสั่งทดสอบไว้ใช้กับแอปพลิเคชัน HomePro E-Catalog ควบคู่กับ AWS Device Farm
        \item จัดทำเอกสารคู่มือการติดตั้ง, การใช้งาน AWS Device Farm, วิธีการพัฒนาชุดคำสั่งทดสอบ
    \end{enumerate}

    \subsection{ขอบเขตของโครงการ}
        จัดทำชุดคำสั่งทดสอบอัตโนมัติกับแอปพลิเคชันที่ถูกสร้างขึ้นมาด้วย Flutter โดยจะสามารถทดสอบในระบบ Android ได้เท่านั้นและเป็นการทดสอบที่เป็นรูปแบบ Test To Pass โดยกรณีศึกษาจากแอปพลิเคชัน HomePro E-Catalog
        โดยสามรถแบ่งการทดสอบเป็น 35 เหตุการณ์โดยสามารถแบ่งดังนี้
        
        \quad 1. การเขียนการทดสอบด้วยการจับ Element บนหน้าจอโดยไม่ต้องแก้ไขที่ Source Code โดยแบ่งเป็นหน้าจอดังนี้
            \begin{itemize}
                \item หน้าจอ Log-in
                \item หน้าจอ ออกจากระบบ
                \item หน้าจอ รายละเอียดสินค้า
                \item หน้าจอ เปรียบเทียบสินค้า
                \item หน้าจอ รถเข็นสินค้า
                \item หน้าจอ หมวดผู้ใช้งาน
                \item หน้าจอ หมวดเมนู
            \end{itemize}

        \quad 2. การเขียนการทดสอบด้วยการจับ แก้ไขที่ Source Code ของ Flutter โดยใช้ Appium Flutter Driver โดยแบ่งตามหน้าจอดังนี้
            \begin{itemize}
                \item หน้าจอ หมวดสินค้า (Level 3)
                \item หน้าจอ หมวดหมู่ย่อย (Level 2)
                \item หน้าจอ หมวดหมู่ย่อย (Level 1)
            \end{itemize}

        เมื่อพัฒนาชุดคำสั่งเสร็จสิ้นจึงนำไปทดสอบบน AWS Device Farm และจัดทำเอกสารคู่มือวิธีการติดตั้ง, วิธีการพัฒนาชุดคำสั่งทดสอบ, คู่มือการใช้งาน AWS Device Farm

\newpage
\section{รายละเอียดของงานที่ปฏิบัตินอกเหนือจากโครงการที่รับผิดชอบ}
    นอกเหนือจากงานโครงการที่ได้รับผิดชอบยังมีงานอื่นในการช่วยการทำงานของแผนกในฐานะ PROGRAMMER โดยสามารถแบ่งโปรเจ็คที่ได้ทำเป็น 2 ประเภทได้แก่
    
    \subsection{บริการระบบงานขาย Single Sale}
        เป็นระบบที่ใช้ในการยืนยันการซื้อขายสินค้าโดยผู้ใช้งานจะเป็นพนักงานของสาขา {\Company} โดยงานที่ได้รับมอบหมายส่วนใหญ่คือการหาข้อผิดพลาดของระบบและทำการแก้ไข
        ยกตัวอย่าง เช่น การนำข้อมูลออกมาแสดงไม่ถูกต้องจึงต้องไปดูวิธีการนำข้อมูลออกมาและทำการแก้ไขให้ทำงานได้อย่างถูกต้อง หรือ ทำการสร้างหมวดย่อยใหม่เป็นประเภทในการสั่งซื้อสินค้าของลูกค้าเป็นต้น
    
    \subsection{ระบบงานจัดส่งและบริการ Delivery Service}
        เป็นระบบที่ใช้ในการสร้างและยืนยันปิดงานจัดส่งสินค้าโดยผู้ใช้งานจะเป็น พนักงานสาขา, พนักงานจัดส่ง, คอลเซ็นเตอร์ ของ {\Company} โดยงานที่ได้รับมอบหมายส่วนใหญ่คือการหาข้อผิดพลาดของระบบและทำการแก้ไข
        ยกตัวอย่าง เช่น การจัดทีมช่างไปที่บ้านลูกค้าแสดงไม่ถูกต้องจึงต้องทำการแก้ไขให้แสดงได้อย่างถูกต้อง หรือ การปิดงานบางครั้งเป็นงานต่อเนื่องทำหลายวัน
        แต่ระบบได้ปิดงานไปแล้วจึงต้องทำการแก้ไขให้สามารถเก็บการปิดงานเป็นรายวันได้

\section{ลักษณะขั้นตอนกํารทำงาน}
        ลักษณะขั้นตอนการทำงานเป็น รูปแบบ WaterFall มี step การทำอย่างชัดเจนโดยสามารถแบ่งการทำงานดังนี้
        \begin{enumerate}
            \item ศึกษาแอปพลิเคชันที่ต้องการนำมาทดสอบ
            \item ศึกษาวิธีการพัฒนาชุดคำสั่งทดสอบอัตโนมัติ
            \item พัฒนาชุดคำสั่งทำสอบอัตโนมัติแล้วนำไปทดสอบบน AWS Device Farm
            \item จัดทำคู่มือวิธีการติดตั้ง, จัดทำคู่มือวิธีการพัฒนาชุดคำสั่งทดสอบอัตโนมัติด้วย, จัดทำคู่มือวิธีการใช้งาน
            \item รายงานผลการทดลอง
            \item ส่งมอบชุดคำสั่งทดสอบ
        \end{enumerate}


\section{ทฤษฎีที่เกี่ยวข้อง}
    \subsection{การทดสอบซอฟต์แวร์ (Software Testing)}
        Software Testing คือ การทดสอบว่าระบบทำงานได้อย่างถูกต้องหรือไม่ตามวัตถุประสงค์หรือเปล่าและสามารถระบุข้อผิดพลาดเพื่อสามารถนำไปแก้ไขได้
        ก่อนการนำไปจัดส่งซึงการทำการทดสอบซอฟต์แวร์นั้นมีความสำคัญมากเนื่องจากการเจอ ข้อผิดพลาดในซอฟต์แวร์นั้นมีค่าใช้จ่ายที่สูงหากเกิดขึ้นตอนนำจัดส่งไปแล้ว
        โดยการทดสอบซอฟต์แวร์แบ่งเป็นได้ 2 ประเภทได้แก่
        \begin{enumerate}
            \item Manual Testing คือ การทดสอบที่ไม่ใช้เครื่องมืออัตโนมัติหรือ Script เลยจะทดสอบตาม Test Plan, Test Case หรือ Test Scenarios ด้วยมือของผู้ทดสอบเอง
            \item Automation Testing คือ การทดสอบอัตโนมัติด้วยการเขียนชุดคำสั่งในการทดสอบ (Script)
        \end{enumerate}
        \begin{figure}[H]
            \centering
            \includegraphics[width=1\textwidth]{cost-of-fixing-bug}
            \caption{ค่าใช้จ่ายการแก้ข้อผิดพลาดที่แปรผันตามขั้นตอนของการพัฒนาซอฟต์แวร์}\label{cost-of-fixing-bug}
        \end{figure}

    \subsection{การทดสอบอัตโนมัติ (Automation Testing)}
        Automation Testing คือ การทดสอบแบบอัตโนมัติโดยการเขียนชุดคำสั่งทดสอบแทนแบบเดิมที่ใช้การทดสอบก้วยมือ ยกตัวอย่าง เช่น
        การทดสอบซอฟต์แวร์แบบเดิมด้วยการใช้มือจะต้องกรอกแบบสอบถามในแอปพลิเคชันในวันแรกและเจอข้อผิดพลาดวันถัดไปนักพัฒนาแอปพลิเคชัน
        ก็ปรับปรุงแอปพลิเคชันมาใหม่ให้ไปทดสอบกรอกแบบเดิมอีกและอาจเจอข้อผิดพลาดใหม่หรือไม่เจอแต่ถ้าหากเกิดการแก้ไขหรือเปลี่ยนแปลงกับ
        ตัวแอปพลิเคชันแล้วต้องทำการทดสอบใหม่อยู่ตลอดซึ่งเป็นการทำงานรูปแบบเดิม แต่การทำ Automation Testing จะมาช่วยแก้ปัญหาโดย
        การเขียนชุดคำสั่งเพื่อมากรอกแบบทดสอบให้ในแอปพลิเคชันซึ่งกำหนดไว้ว่าสิ่งที่ถูกต้องควรจะเป็นอย่างไรและหากไม่ถูกต้องไม่ถูกต้องอย่างไร
        โดยจะเป็นการทำแบบอัตโนมัติ ดังนั้นข้อดีของ Automation Testing ได้แก่
        \begin{itemize}
            \item[-] ผลตอบรับที่ไวขึ้นต่อรอบการพัฒนาหรือแก้ไขแอปพลิเคชัน
            \item[-] สามารถประหยัดค่าใช้จ่ายในการทดสอบได้
            \item[-] สามารถทดสอบได้อย่างคลอบคลุมมากขึ้น
            \item[-] สามารถนำแอปพลิเคชันมาส่งมอบได้เร็วขึ้น
            \item[-] เพิ่มความแม่นยำในการทดสอบมากขึ้น 
            \item[-] กำจัดการทดสอบที่จะผิดพลาดที่เกิดจากมนุษย์ 
        \end{itemize}
        ในปัจจุบันมีเครื่องมือช่วยในการทำ Automation Testing มากมายยกตัวอย่างดังนี้
        \begin{enumerate}

            \item Katalon Studio คือ เครื่องมือตัวหนึ่งในการช่วยทำ Test Automation ของ Mobile Applications ซึ่งสามารถทดสอบได้ทั้ง
            Android และ IOS
            \begin{figure}[H]
                \centering
                \includegraphics[width=0.3\textwidth]{katalon-studio}
                \caption{ตราตราสัญลักษณ์ Katalon Studio}\label{katalon-studio}
            \end{figure}

            \item Selenium คือ Software Testing Framework ที่มีประสิทธิภาพไว้ใช้สำหรับเขียนชุดคำสั่งทดสอบ Web Applications ซึ่งเป็นแบบ Open Source สามารถเขียนได้ด้วยหลายภาษา เช่น Java, Python, \texttt(C\#), Javascript, PHP, Perl 
            \begin{figure}[H]
                \centering
                \includegraphics[width=0.5\textwidth]{selenium}
                \caption{ตราตราสัญลักษณ์ Selenium}\label{selenium}
            \end{figure}

            \item Micro Focus UFT คือ หนึ่งใน Software ที่มีประสิทธิภาพที่สุดสำหรับการทำการทดสอบแบบ Functional Testing สามารถสร้าง Test และแก้ไขได้อย่างรวดเร็วไปจนถึงสามารถนำเทคโนโลยี Object Recognition, Image-based Automation และ Machine Driven Regression Testing เข้ามาใช้ช่วยในการทำงาน
            แต่เสียค่าใช้จ่ายแต่มีให้ทดลองใช้งานฟรี 60 วัน
            \begin{figure}[H]
                \centering
                \includegraphics[width=0.20\textwidth]{uft}
                \caption{ตราตราสัญลักษณ์ Micro Focus UFT}\label{uft}
            \end{figure}

            \item TestComplete คือ หนึ่งใน Software ที่มีประสิทธิภาพที่สุดสำหรับการทำการทดสอบ Desktop, Mobile และ Web Applications ตามชุดคำสั่งที่เขียนได้ด้วย
            ภาษา Python, JavaScript, VBScript และอื่นๆ
            \begin{figure}[H]
                \centering
                \includegraphics[width=0.5\textwidth]{test-com}
                \caption{ตราตราสัญลักษณ์ TestComplete}\label{test-com}
            \end{figure}


        \end{enumerate}

    \subsection{Appium}
        Appium คือ เครื่องมือสำหรับการทำ Automation Testing เป็นรุปแบบ Open Source ไว้สำหรับทดสอบ Native, Mobile Web, Hybrid
        , Android, IOS และ Windows Desktop การใช้ Appium จะเป็น Cross Platform หมายความว่าจะทำให้สามารถเขียน
        โดยใช้ API เดียวกันซึ่งจะช่วยให้สามารถใช้โค้ดซ้ำระหว่างอุปกรณ์ที่ทดสอบได้ IOS, Android, Window การทำงานจะสื่อสารระหว่าง Driver กับ Appium ผ่าน JSON โดยรองรับการเขียนได้หลายภาษา
        เช่น Python, Java, JavaScript(NodeJS), Ruby 
        \begin{figure}[H]
            \centering
            \includegraphics[width=0.3\textwidth]{appium}
            \caption{ตราตราสัญลักษณ์ Appium}\label{appium}
        \end{figure}

        Driver ที่สามารถใช้กับ Appium ได้แก่
        \begin{itemize}
            \item XCUITest Driver (for iOS and tvOS apps)
            \item Espresso Driver (for Android apps)
            \item UiAutomator2 Driver (for Android apps)
            \item Windows Driver (for Windows Desktop apps)
            \item Mac Driver (for Mac Desktop apps)
        \end{itemize}

        \begin{figure}[H]
            \centering
            \includegraphics[width=1\textwidth]{appium-arc}
            \caption{โครงสร้างการทำงานของ Appium}\label{appium-arc}
        \end{figure}

        นอกเหนือจากนี้ Appium ยังสามารถใช้งานร่วมกับ AWS Device Farm ได้

    \subsection{API}
        API (Application Programming Interface) คือ วิธีการติดต่อสื่อสารระหว่างแอปพลิเคชันไม่ว่าแอปพลิเคชันนั้นจะรันอยู่บนอุปกรณ์ใด เช่นคอมพิวเตอร์ โทรศัพท์มือถือ หรือเฟิร์มแวร์ในอุปกรณ์เครื่องใช้ต่างๆ โดยที่แอปพลิเคชันฝั่งหนึ่งเป็นผู้ขอใช้บริการหรือขอข้อมูลจากแอปพลิเคชันอีกฝั่งหนึ่งซึ่งเป็นผู้ให้บริการ การติดต่อสื่อสารระหว่างแอปพลิเคชันดังกล่าวเป็นไปโดยอัตโนมัติตามที่ได้กำหนดไว้

    \subsection{AWS Device Farm}
         AWS คือ Amazon Web Services เป็นคลาวด์แพลตฟอร์มที่มีคนนำมาใช้มากที่สุดในโลกที่มีการบริการ 175 บริการ
         โดยองค์กรขนาดใหญ่หรือสตาร์ทอัพก็เริ่มหันมาใช้ AWS เพื่อลดค่าใช้จ่ายและความคล่องตัว

        \begin{figure}[H]
            \centering
            \includegraphics[width=0.5\textwidth]{aws}
            \caption{ตราสัญลักษณ์ Amazon Web Service (AWS)}\label{aws}
        \end{figure}

        AWS Device Farm คือ บริการหนึ่งของ AWS เป็นบริการไว้ทดสอบแอปพลิเคชันเพื่อปรับปรุงคุณภาพแอปพลิเคชันหรือระบบต่างๆ
        โดย AWS Device Farm จะทดสอบแอปพลิเคชันหรือระบบใน Desktop, Browser หรืออุปกรณ์มือถือที่หลากหลายทั้งในระบบปฎิบัติการ Android และ IOS พร้อมกันเพื่อช่วย
        ให้ชุดทดสอบรวดเร็วขึ้น หลากหลายมากขึ้น และพร้อมสร้างวีดีโอและบันทึกเพื่อช่วยให้หาปัญหาของระบบหรือแอปพลิเคชันได้ไวยิ่งขึ้น

        \begin{figure}[H]
            \centering
            \includegraphics[width=0.25\textwidth]{amazon-device-farm}
            \caption{ตราสัญลักษณ์ AWS Device Farm}\label{amazon-device-farm}
        \end{figure}

    \subsection{Node.Js}
        Node.Js คือ JavaScript runtime environment เป็น OpenSource คือการที่สามารถนำเอา JavaScript มาใช้งานแบบภาษาอื่นบน
        Windows, Linux หรือ Mac ได้แบบไม่เสียค่าใช้จ่ายหากติดตั้ง Node.js จะสามารถเขียนโปรแกรมด้วยภาษา JavaScript เหมือนกับ Java, \texttt(C\#), Python
        ซึ่งหลักๆแล้วจะนำมาทำเป็น backend server นอกจากนี้ Node.Js ยังมี NPM (Node Package Manager) เป็นตัวที่ใช้สำหรับการดาวน์โหลด
        library ภายนอกมาใช้โดยติดตั้งเพียงพิมพ์ `npm install <ชื่อ library>` เช่น mocha, express, chai เป็นต้น

        \begin{figure}[H]
            \centering
            \includegraphics[width=0.3\textwidth]{node}
            \caption{ตราสัญลักษณ์ Node.JS}\label{node}
        \end{figure}
        

    \subsection{Flutter}
        Flutter คือ Framework แบบ OpenSource ที่ถูกพัฒนาโดย Google มีไว้เพื่อใช้สร้าง UserInterface สำหรับ Mobile Application ที่สามารถทำงานข้ามแพลตฟอร์มได้ทั้ง IOS และ Android คือเขียนโปรแกรมหนึ่งครั้งสามารถนำมาใช้ได้ทั้งสองแพลตฟอร์มโดยภาษาที่ Flutter ใช้คือภาษา Dart
        โดยจุดเด่นของ Flutter คือระบบ Hot Reload จะเข้ามาช่วยในส่วนของการ reload สามารถพัฒนาแอปพลิเคชัน ในส่วน UserInterface มีความรวดเร็วมากขึ้นอีกทั้งยังมีความสวยงามแบบ
        Material Design

        \begin{figure}[H]
            \centering
            \includegraphics[width=0.5\textwidth]{flutter}
            \caption{ตราสัญลักษณ์ Flutter}\label{flutter}
        \end{figure}

    \subsection{Appium Flutter Driver}
        Appium Flutter Driver คือ เครื่องมือช่วยในการทำ Automation Test กับแอปพลิเคชันที่สร้างมาจาก Flutter เป็นส่วนหนึ่งในการใช้งานกับ Appium โดย Appium Flutter Driver จะใช้ Dart Service Protocol เพื่อส่ง API ไปเรียกใช้การ Test ของ Flutter ที่ทั่วไปต้องเขียนเป็นภาษา Dart แต่ถ้าใช้ library นี้จะเขียนภาษาตามที่ Appium มีได้เลย

        \begin{figure}[H]
            \centering
            \includegraphics[width=1\textwidth]{appium-flutter}
            \caption{โครงสร้างการทำงานของ Appium Flutter Driver}\label{appium-flutter}
        \end{figure}

    \subsection{"Wd"}
        Wd คือ library JavaScript ที่ใช้ในการทำ Automation Test ใน NodeJS โดยที่สามารถทำงานร่วมกับ Selenium และ Appium ใช้จับ element แบบ xpath

    \subsection{WebdriverIO}
        WebdriverIO คือ library JavaScript ที่ใช้ในการทำ Automation Test ใน NodeJS โดยที่สามารถทำงานร่วมกับ Selenium และ Appium ได้

    \subsection{Xpath}
        Xpath คือ ตัวชี้ทางในภาษาต่างๆ (เช่น XML, HTML) เพื่อแสดง Root ของเส้นทางการเข้าถึงข้อมูลตามลำดับชั้น
        
    \subsection{Mocha}
        Mocha คือ library JavaScript ที่ใช้ใน NodeJs เพื่อทำการทดสอบอัตโนมัติแบบ Asynchronous Testing ได้ง่ายขึ้นโดยการแสดงผลลัพธ์ที่ผิดพลาดอย่างง่ายและชัดเจนตาม Test Case

    \subsection{Chai}
        Chai คือ library JavaScript ที่ใช้ใน NodeJs ทำหน้าที่เปรียบที่ค่าผลลัพธิ์ที่ได้จากการทดสอบกับผลลัพธ์ที่ควรจะเป็นโดยเป็นรูปแบบที่เข้าใจง่าย

    \subsection{Git}
        Git คือ Vesion Control เป็นตัวที่ใช้จัดเก็บและคอยดูการเปลี่ยนแปลงกับไฟล์ชนิดใดก็ได้เมื่อจัดเก็บไฟล์เข้าไปในระบบของ Git แล้วจะเรียกว่า Git Repository ซึ่งสำรองข้อมูลของ Source Code สามารถย้อนกลับไปเวอร์ชั่นใดก่อนหน้าและดูรายละเอียดการเปลี่ยนแปลงของแต่ละเวอร์ชั่นได้

        \begin{figure}[H]
            \centering
            \includegraphics[width=0.3\textwidth]{git-logo}
            \caption{ตราสัญลักษณ์ Git}\label{git-logo}
        \end{figure}

    \subsection{Visual Studio Code}
        Visual Studio Code คือ Editor ตัวหนึ่งที่สร้างมาเพื่ออำนวยความสะดวกแก่โปรแกรมเมอร์มีธีมและรองรับรูปแบบการเขียนได้หลายภาษาอีกทั้งยังมีตัวช่วยในการเขียนโปรแกรมต่างๆ เช่น Bracket Matcher, Live Server เป็นต้น

        \begin{figure}[H]
            \centering
            \includegraphics[width=0.2\textwidth]{visual-studio-code}
            \caption{ตราสัญลักษณ์ Visual Studio Code}\label{visual-studio-code}
        \end{figure}
    \chapter{การออกแบบระบบ และรายละเอียดการพัฒนา}
\thispagestyle{empty}
\label{chapter:designAndDevelop}

\titleformat{\paragraph}
{\normalfont\normalsize\bfseries}{\theparagraph}{1em}{}
\titlespacing*{\paragraph}
{0pt}{3.25ex plus 1ex minus .2ex}{1.5ex plus .2ex}

% \section{ภาพรวมของระบบ}
% ส่วนการจองซื้อหุ้นกู้เป็นส่วนหนึ่งของแอปพลิเคชัน ซีไอเอ็มบีไทยดิจิตอลแบงก์กิ้ง ซึ่งมีแนวคิดเพื่อทำให้ลูกค้าทำธุรกรรมทางการเงินได้ง่ายขึ้น
% ไม่ว่าจะเป็นการโอนเงินไปยังธนาคารต่าง ๆ หรือการโอนเงินบัญชีพร้อมท์เพย์ การตรวจสอบบัญชี การเปิดบัญชี E-Saving ผ่านแอปพลิเคชัน
% การตรวจสอบบัญชีสินเชื่อต่าง ๆ อีกทั้งยังมีเรื่องการลงทุนที่ธนาคาร ซีไอเอ็มบีไทย จำกัด (มหาชน) 
% ไม่ได้มองข้ามโดยในส่วนการจองซื้อหุ้นกู้ถูกพัฒนาเพื่อเพิ่มช่องทาง และยังเป็นการเพิ่มความสะดวกแก่ลูกค้า อีกทั้งยังเป็นการเพิ่มรายได้ให้แก่ธนาคารอีกด้วย
% โดยหุ้นกู้จะถูกจัดการข้อมูลผ่าน Web Operation Tool ที่เป็นเครื่องมื่อที่ใช้จัดการข้อมูลต่าง ๆ ในแอปพลิเคชัน ซีไอเอ็มบีไทยดิจิตอลแบงก์กิ้ง

% \section{โครงสร้างของระบบ}
% การพัฒนาส่วนการจองซื้อหุ้นกู้สำหรับแอปพลิเคชัน ซีไอเอ็มบีไทยดิจิตอลแบงก์กิ้ง มีการพัฒนาทั้งบนระบบปฏิบัติการ Android โดยใช้ภาษา Kotlin
% และบนระบบปฏิบัติการ iOS โดยใช้ภาษา Swift อีกทั้งยังมีการใช้รูปแบบการพัฒนาแอปพลิเคชันแบบ Super App และ Mini App
% เพื่อที่จะทำให้แต่ละทีมสามารถพัฒนาไปพร้อมกันได้โดยไม่เกิดการขัดกัน โดยส่วนต่าง ๆ ในแอปพลิเคชันจะเป็น Mini App และจะถูกรวบรวมเอาไว้ที่
% Super App เพื่อเป็นแอปพลิเคชันที่สมบูรณ์ดังรูปที่ \ref{super-app-mini-app} โดย Super App จะมอง Mini App ทั้งหมดเป็น Dependencies 
% ที่ Super App ต้องเรียกใช้เพื่อให้แอปพลิเคชันทำงานได้โดยแต่ละ Mini App ก็ต้อง Export ตัวเองออกมาให้เป็น Library ซึ่งแต่ละทีมแยกกันทำงานได้
% โดยไม่ขัดกันดังที่กล่าวไว้ข้างต้น
% \begin{figure}[H]
%     \centering
%     \includegraphics[width=1\textwidth]{super-app-mini-app}
%     \caption{โครงสร้างการพัฒนาแอปพลิเคชันรูปแบบ Super App และ Mini App}\label{super-app-mini-app}
% \end{figure}

% ส่วนเว็บไซต์การจัดการข้อมูลหุ้นกู้ใช้ภาษา Python ในการพัฒนาโดยใช้ประโยชน์จาก Django Framework เข้า
% มาแบ่งแยกส่วนต่าง ๆ เพื่อให้ตอบสนองการทำงานเป็นทีมได้ดีขึ้น และยังมีส่วนของภาษา JavaScript ที่ใช้ประโยชน์จาก Vue.js
% เข้ามาเป็นตัวควบคุมข้อมูลกับ API และสุดท้ายในส่วนของ API นั้นพัฒนาด้วยภาษา Java ซึ่งใช้ประโยชน์จาก Spring Boot Framework
% โดยมีโครงสร้างระบบดังรูปที่ \ref{app-structure}
% \begin{figure}[H]
%     \centering
%     \includegraphics[width=1\textwidth]{app-structure}
%     \caption{โครงสร้างระบบการจองซื้อหุ้นกู้ในแอปพลิเคชัน ซีไอเอ็มบีไทยดิจิตอลแบงก์กิ้ง}\label{app-structure}
% \end{figure}


% \section{คุณสมบัติหลัก}
% เมื่อได้รับความต้องการจากหน่วยธุรกิจมาแล้ว ทางทีมพัฒนาจึงได้เริ่มวางแผนในส่วนต่าง ๆ ที่ต้องมีในระบบดังนี้
% \begin{enumerate}
%     \item \textbf{Term and Conditions}\newline เมื่อลูกค้าในแอปพลิเคชันและเข้ามาในส่วนการจองซื้อหุ้นเป็นครั้งแรก ลูกค้าต้องสามารถกดยอมรับข้อกำหนดการใช้งาน(Term And Conditions)ได้
%     \item \textbf{Bond Home Screen}\newline โดยลูกค้าเข้ามายังส่วนการจองซื้อหุ้นกู้แล้วสามารถเลือกได้ระหว่างจองซื้อหุ้นกู้กับประวัติการจองซื้อหุ้นกู้
%     \item \textbf{Bond Transaction History Screen}\newline ลูกค้าสามารถดูรายการประวัติการจองซื้อหุ้นกู้ได้โดยสามารถเห็น จำนวนเงินที่จองซื้อ, วันที่ทำการจองซื้อ, ระดับความน่าเชื่อถือ, อัตราดอกเบื้ย เป็นต้น
%     \item \textbf{Bond Detail Screen}\newline ลูกค้าสามารถดูรายละเอียดทั้งหมดของหุ้นกู้ได้ และสามารถจองซื้อหุ้นกู้ได้ผ่านหน้านี้
%     \item \textbf{CSA Questionnaire Screen}\newline เมื่อลูกค้าการประเมินความเสี่ยงหมดอายุ ลูกค้าจำเป็นต้องทำแบบประเมินเพื่อที่จะได้รู้ระดับของความเสี่ยงว่ารับความเสี่ยงของหุ้นกู้ได้ระดับไหนโดยระดับมีตั้งแต่ 0 - 8
%     \item \textbf{Factsheet Screen}\newline เมื่อลูกค้าต้องการจองซื้อหุ้นกู้ลูกค้าจำเป็นต้องอ่านหนังสือสรุปข้อมูลสำคัญของหุ้นกู้ แล้วกดยอมรับเพื่อจองซื้อหุ้นกู้
%     \item \textbf{Warning List (Risk Acceptance) Screen}\newline ลูกค้าจำเป็นต้องรู้ความเสี่ยงต่าง ๆ ที่ได้รับเมื่อต้องการซื้อหุ้นกู้ ซึ่งขึ้นอยู่กับเงื่อนไขต่าง ๆ เช่น ความเสี่ยงทั่วไปเมื่อต้องการจองซื้อหุ้นกู้ ความเสี่ยงที่ระดับความเสี่ยงของลูกค้าน้อยกว่าระดับความเสี่ยงของหุ้นกู้ และในกรณีที่ลูกค้าเป็นลูกค้าเปราะบางซึ่งลูกค้าต้องกดยอมรับความเสี่ยงทั้งหมดจึงจะสามารถจองซื้อหุ้นกู้ได้
%     \item \textbf{Payment Confirmation Screen}\newline ลูกค้าสามารถเลือกบัญชีที่ใช้การหักเงินได้ สามารถใส่จำนวนเงินที่ต้องการซื้อโดยขั้นต่ำเท่ากับราคาพาร์คูณกับจำหน่วยการลงทุนขั้นต่ำ สามารถกดเพิ่มลดจำนวนเงินตามราคาทวีคูณได้ และสามารถจดบันทึกรายละเอียดการจองซื้อหุ้นกู้ได้
%     \item \textbf{Benificial Information Screen}\newline ลูกค้าสามารถเลือกได้ว่าจะรับผลประโยชน์จากหุ้นกู้ทางใดไม่ว่าจะเป็นเช็ค หรือบัญชี ซึ่งถ้าเลือกบัญชีลูกค้าสามารถเลือกได้ว่าจะรับผลประโยชน์ด้วยบัญชีใด
%     \item \textbf{Receive Bond Information Screen}\newline ลูกค้าสามารถเลือกได้ว่าจะรับเอกสารหุ้นกู้ได้ระหว่างไปรษณีย์ หรือศูนย์รับฝาก ซึ่งในกรณีไปรษณีย์ลูกค้าสามารถแก้ไขที่อยู่ได้ และสามารถแก้ไขอีเมลได้ ส่วนศูนย์รับฝากสามารถเลือกศูนย์รับฝากจากรายการได้ สามารถใส่เลขที่ศูนย์รับฝากได้ และแก้ไขที่อยู่ได้ โดยลูกค้าสามารถกดจองซื้อหุ้นกู้จากหน้านี้ได้
%     \item \textbf{Review Bond Booking detail Screen}\newline ลูกค้าสามารถเห็นรายละเอียดการจองซื้อหุ้นกู้ได้ทั้งหมดและสามารถกดยืนยันรายการสั่งซื้อจากหน้านี้ได้
%     \item \textbf{Passcode Screen}\newline เมื่อกดยืนยันรายการสั่งซื้อต้องขึ้นหน้าให้ใส่รหัสผ่านของแอปพลิเคชันด้วยเพื่อยืนยันตัวตนของลูกค้าก่อนจะหักเงินบัญชี
%     \item \textbf{Receipt Screen}\newline เมื่อทำรายการสั่งซื้อเสร็จสิ้น ลูกค้าสามารถเห็นใบเสร็จสั่งซื้อ และสามารถบันทึกรูปภาพของใบเสร็จสั่งซื้อลงอัลบั้มภาพของโทรศัพท์มือถือได้อัตโนมัติ
% \end{enumerate}

% \newpage
% \section{ขอบเขตงานที่นักศึกษารับผิดชอบ}
%     นักศึกษานั้นได้อยู่ในทีม Treasury ของแผนก Digital Banking ได้มีส่วนรวมในการพัฒนาแอปพลิเคชันในส่วนการจองซื้อหุ้นกู้บนระบบปฏิบัติการ Android รวมไปถึงพัฒนาหน้า Bond Transaction History Screen 
%     บนระบบปฏิบัติการ iOS ตลอดจนทำชุดทดสอบอัตโนมัติสำหรับส่วนติดต่อลูกค้าทั้งส่วนของ Android และ iOS เพื่อการทดสอบบน Jenkins อีกด้วย

% \section{วิธีการพัฒนาระบบ}
% \subsection{Product Backlog Grooming}
% โดยทุก ๆ วันแรกของสปริ้นท์นักพัฒนาทุกคนรวมไปถึง Product Owner และ Project Manager จะมาทำความเข้าใจใน Epic และ Issue ที่มีทั้งหมดใน Sprint
% เพื่อที่จะได้ทำตามความต้องการทางธุรกิจซึ่งได้อิงตาม Epic ที่เป็นรายละเอียดความต้องของทางหน่วยธุรกิจที่ Product Owner ได้กำหนดไว้ ซึ่ง Issue 
% เกิดขึ้นจากหัวหน้านักพัฒนาได้นำ Epic มาแยกเป็น User Story ซึ่งเป็นหนึ่งในชนิดของ Issue โดยรายละเอียดของ Epic มีดังตารางที่ \ref{tableOfTask} 
% \begin{tabularx}{\linewidth}{|c|X|c|c|c|}
% 	\caption{รายละเอียดของ Epic}\label{tableOfTask} \\
% 	\hline
%     \multicolumn{1}{|c|}{\textbf{Epic ID}}	&	\multicolumn{1}{c|}{\textbf{Epic}} &	\multicolumn{1}{c|}{\textbf{Total Issue}} &	\multicolumn{1}{c|}{\textbf{Total Story Points}}	&   \multicolumn{1}{c|}{\textbf{Sprint}} \\
% 	\hline
% 	\endfirsthead
% 	\caption* {\textbf{ตารางที่ \ref{tableOfTask} (ต่อ)} รายละเอียดของ Epic} \\
% 	\hline
% 	\multicolumn{1}{|c|}{\textbf{Epic ID}}	&	\multicolumn{1}{c|}{\textbf{Epic}} &	\multicolumn{1}{c|}{\textbf{Total Issue}} &	\multicolumn{1}{c|}{\textbf{Total Story Points}}	 &	\multicolumn{1}{c|}{\textbf{Sprint}} \\
% 	\hline
% 	\endhead
% 	\hline
% 	\endfoot
% 	DA-11724 &Treasury - Product Infomation & 26 & 16 & 20\\
% 	DA-10811 &Treasury - Customer Segment & 28 & 24 & 21\\
% 	DA-11729 &Treasury - Booking & 69 & 47 & 22 - 23\\
% 	\hline
% \end{tabularx}

% นอกเหนือจากการอธิบายรายละเอียดของ Epic และ Issue ให้เข้าใจนั้น ยังมีการประเมินค่าคะแนนของ Issue ที่นักพัฒนาทุกคนต้องร่วมกันประเมินว่าแต่ละ Issue
% มีความยากง่ายแค่ไหน ใช้เวลาในการดำเนินการเท่าไร โดยการประเมินค่าคะแนนจะใช้ลำดับฟีโบนัชชีในการประเมินเพื่อให้คะแนนมีความแตกต่างกันอย่างชัดเจน
% เมื่อ Issue นั้นไม่สามารถหาคะแนนลงได้เนื่องจากว่าอาจจะใหญ่เกินไป ทำให้หัวหน้านักพัฒนาสามารถแยก Issue ให้เล็กกว่านี้ได้อีก ซึ่งการให้คะแนนจะทำไปเรื่อย ๆ
% จนกว่าทุกคนจะให้คะแนนเท่ากัน และเข้าใจเหตุผลของการให้คะแนนโดยรายละเอียดของ Issue ของแต่ละ Epic มีรายละเอียดดังตารางที่ \ref{tableOfStoryProductInfo}, \ref{tableOfStoryCustomerSeg}, \ref{tableOfStoryBookingSP22} และ \ref{tableOfStoryBookingSP23}
% \begin{tabularx}{\linewidth}{|c|X|c|c|}
% 	\caption{รายละเอียดของ Issue ของ Treasury - Product Infomation}\label{tableOfStoryProductInfo} \\
% 	\hline
%     \multicolumn{1}{|c|}{\textbf{Issue ID}}	&	\multicolumn{1}{c|}{\textbf{Issue}} &	\multicolumn{1}{c|}{\textbf{Issue Type}}  &	\multicolumn{1}{c|}{\textbf{Story Point}} \\
% 	\hline
% 	\endfirsthead
% 	\caption* {\textbf{ตารางที่ \ref{tableOfStoryProductInfo} (ต่อ)} รายละเอียดของ Issue ของ Treasury - Product Infomation} \\
% 	\hline
% 	\multicolumn{1}{|c|}{\textbf{Issue ID}}	&	\multicolumn{1}{c|}{\textbf{Issue}} &	\multicolumn{1}{c|}{\textbf{Issue Type}}  &	\multicolumn{1}{c|}{\textbf{Story Point}} \\
% 	\hline
% 	\endhead
% 	\hline
% 	\endfoot
% 	DA-12630 &[iOS][Success] Customer can see menu 'หุ้นกู้' in mini app menu &Story &1\\
% 	DA-12640 &[iOS][Success] Customer can navigate to detail and then can back to ProductList. &Story &1\\
%     DA-12606 &[iOS][Success] Customer can see T\&C when go to bond first time &Story &2\\
%     DA-12645 &[iOS][Fail] Customer can see error dialog when has error on T\&C. &Story &1\\
%     DA-12607 &[iOS][Success] Customer can see 'รายการหุ้นกู้' (bond product list) &Story &3\\
%     DA-12609 &[iOS][Success] Customer can see empty page when bond list is empty &Story &2\\
%     DA-12608 &[iOS][Fail] Customer can see error dialog when go to product list and any error occurred &Story &1\\
%     DA-12610 &[iOS][Success] On booking history tab : Customer should see empty booking history list. &Story &1\\
%     DA-12627 &[Android][Success] Customer can see 'รายละเอียดหุ้นกู้' (bond product detail) &Story &2\\
%     DA-12629 &[Android][Success] Customer can see prospectus when click 'เอกสารสรุปข้อมูลสำคัญ' in bond detail page &Story &2\\
%     DA-12628 &[Android][Success] Customer can see factsheet when click 'หนังสือชี้ชวน' in bond detail page &Story &2\\
%     DA-12615 &[Web][Success] Admin can add bond information &Story &2\\
%     DA-12616 &[Web][Success] Admin can view bond information &Story &2\\
%     DA-12617 &[Web][Success] Admin can edit bond information &Story &2\\
%     DA-12618 &[Web][Success] Admin can delete bond information &Story &2\\
%     DA-12619 &[Web][Success] Prevent XSS &Story &1\\
%     DA-14547 &[iOS][Success] Customer can see product name on the product list and product detail &Story &1\\
%     DA-14548 &[Android][Success] Customer can see product name on the product list and product detail &Story &1\\
% 	\hline
% \end{tabularx}

% จากตารางที่ \ref{tableOfStoryProductInfo}  นี้จะเป็น Issue ทั้งหมดของ Epic Treasury - Product Info. ที่เกี่ยวกับหน้ารายการหุ้นกู้ และรายละเอียดหุ้นกู้ โดยมี Web Operation Tool เป็นตัวจัดการข้อมูลทั้งเพิ่ม
% ลบ และแก้ไข โดยสามารถกดเข้าไปดูรายละเอียดหุ้นกู้แต่ละรายการได้ ซึ่งทีม Treasury ได้รับมอบหมายให้ดูส่วนของรายละเอียดหุ้นกู้ และเว็บไซต์เพื่อจัดการข้อมูล

% \clearpage
% \begin{tabularx}{\linewidth}{|c|X|c|c|}
% 	\caption{รายละเอียดของ Issue ของ Treasury - Customer Segment}\label{tableOfStoryCustomerSeg} \\
% 	\hline
%     \multicolumn{1}{|c|}{\textbf{Issue ID}}	&	\multicolumn{1}{c|}{\textbf{Issue}} &	\multicolumn{1}{c|}{\textbf{Issue Type}}  &	\multicolumn{1}{c|}{\textbf{Story Point}} \\
% 	\hline
% 	\endfirsthead
% 	\caption* {\textbf{ตารางที่ \ref{tableOfStoryCustomerSeg} (ต่อ)} รายละเอียดของ Issue ของ Treasury - Customer Segment} \\
% 	\hline
% 	\multicolumn{1}{|c|}{\textbf{Issue ID}}	&	\multicolumn{1}{c|}{\textbf{Issue}} &	\multicolumn{1}{c|}{\textbf{Issue Type}}  &	\multicolumn{1}{c|}{\textbf{Story Point}} \\
% 	\hline
% 	\endhead
% 	\hline
% 	\endfoot
% 	DA-13567 &[API] Migrate factsheet url from bond information to legal-document. &Story &2\\
% 	DA-13388 &[iOS][Success] Customer can see factsheet when bond type is PO and customer type is PO &Story &1\\
%     DA-13388 &[iOS][Success] Customer can see factsheet when bond type is PO and customer type is HNW &Story &1\\
%     DA-13389 &[iOS][Success] Customer can see factsheet when bond type is PO and customer type is UHNW &Story &1\\
%     DA-13414 &[Android][Success] Customer can see factsheet when bond type is PO and customer type is PO &Story &1\\
%     DA-13415 &[Android][Success] Customer can see factsheet when bond type is PO and customer type is HNW &Story &1\\
%     DA-13416 &[Android][Success] Customer can see factsheet when bond type is PO and customer type is UHNW &Story &1\\
%     DA-13399 &[iOS][Success] Customer can see eligible list &Story &2\\
%     DA-13425 &[Android][Success] Customer can see eligible list &Story &2\\
%     DA-13400 &[iOS][Success] Customer can accept eligible. &Story &1\\
%     DA-13426 &[Android][Success] Customer can accept eligible. &Story &1\\
%     DA-13401 &[iOS][Success] Customer can see eligible detail &Story &1\\
%     DA-13427 &[Android][Success] Customer can see eligible detail&Story &1\\
%     DA-13438 &[WebOperation][Success] PO can create questionnaire for self declare. &Story &1\\
%     DA-13533 &[Web][Success] [API] Return data from ivp &Story &1\\
%     DA-13432 &[Android][Fail] Customer can see error dialog when can't get eligible legal document &Story &1\\
%     DA-13406 &[iOS][Fail] Customer can see error dialog when can't get eligible legal document &Story &1\\
%     DA-13407 &[iOS][Fail] Customer can see error dialog when can't accept eligible legal document &Story &1\\
%     DA-13433 &[Android][Fail] Customer can see error dialog when can't accept eligible legal document &Story &1\\
%     DA-13402 &[iOS][Fail] Customer can see error dialog when can't get factsheet legal document &Story &1\\
%     DA-13428 &[Android][Fail] Customer can see error dialog when can't get factsheet legal document &Story &1\\
%     DA-13403 &[iOS][Fail] Customer can see error dialog when can't accept factsheet legal document &Story &1\\
%     DA-13429 &[Android][Fail] Customer can see error dialog when can't accept factsheet legal document &Story &1\\
%     DA-14204 &[Backend] Prepare api booking primary bond. &Task &2\\
% 	\hline
% \end{tabularx}

% จากตารางที่ \ref{tableOfStoryCustomerSeg}  นี้จะเป็น Issue ทั้งหมดของ Epic Treasury - Customer Segment ที่เกี่ยวกับส่วนของการส่วนการยืนยันตัวตนเพื่อระบุลูกค้าว่าเป็นประเภทใด ได้แก่ ผู้ลงทุนรายใหญ่, ผู้ลงทุนรายใหญ่ และผู้ลงทุนรายใหญ่พิเศษ
% เมื่อลูกค้ายืนยันตัวตนสำเร็จแล้วลูกค้าจะต้องอ่านเอกสารสรุปข้อมูลสำคัญของหุ้นกู้แล้วยอมรับ เมื่อลูกค้ายืนยันเอกสารสรุปข้อมูลสำคัญหุ้นกู้แล้ว ลูกค้าต้องยอมรับความเสี่ยงของการจองซื้อหุ้นกู้ได้
% โดยทีม Treasury ได้รับผิดชอบในส่วนของเอกสารสรุปข้อมูลสำคัญ และการยอมรับรายการความเสี่ยงของลูกค้า


% \begin{tabularx}{\linewidth}{|c|X|c|c|}
% 	\caption{รายละเอียดของ Issue ของ Treasury - Booking (Sprint 22)}\label{tableOfStoryBookingSP22} \\
% 	\hline
%     \multicolumn{1}{|c|}{\textbf{Issue ID}}	&	\multicolumn{1}{c|}{\textbf{Issue}} &	\multicolumn{1}{c|}{\textbf{Issue Type}}  &	\multicolumn{1}{c|}{\textbf{Story Point}} \\
% 	\hline
% 	\endfirsthead
% 	\caption* {\textbf{ตารางที่ \ref{tableOfStoryBookingSP22} (ต่อ)} รายละเอียดของ Issue ของ Treasury - Booking (Sprint 22)} \\
% 	\hline
% 	\multicolumn{1}{|c|}{\textbf{Issue ID}}	&	\multicolumn{1}{c|}{\textbf{Issue}} &	\multicolumn{1}{c|}{\textbf{Issue Type}}  &	\multicolumn{1}{c|}{\textbf{Story Point}} \\
% 	\hline
% 	\endhead
% 	\hline
% 	\endfoot
% 	DA-14907 &[iOS][Success] Customer can see bond detail and check see suitability (CSA) expired &Story &1\\
% 	DA-14910 &[Android][Success] Customer can see bond detail and check see suitability (CSA) expired &Story &1\\
%     DA-14912 &[iOS][Fail] Customer can see bond detail and check see suitability (CSA) expired when get customer profile risk service not available &Story &1\\
%     DA-14911 &[Android][Fail] Customer can see bond detail and check see suitability (CSA) expired when get customer profile risk service not available. &Story &1\\
%     DA-14529 &[iOS][Success] Customer can see suitability test expired (CSA). &Story &1\\
%     DA-14809 &[Android][Success] Customer can see suitability test expired (CSA). &Story &1\\
%     DA-14530 &[iOS][Success] Customer can see and accept T\&C for suitability test (CSA). &Story &1\\
%     DA-14810 &[Android][Success] Customer can see and accept T\&C for suitability test (CSA). &Story &1\\
%     DA-14531 &[iOS][Fail] Customer can see error dialog when get T\&C not available. &Story &1\\
%     DA-14811 &[Android][Fail] Customer can see error dialog when get T\&C not available. &Story &1\\
%     DA-14533 &[iOS][Fail] Customer can see error dialog when accept T\&C not available. &Story &1\\
%     DA-14820 &[Android][Fail] Customer can see error dialog when accept T\&C not available. &Story &1\\
%     DA-14534 &[iOS][Success]Customer can see suitability test (CSA). &Story &3\\
%     DA-14821 &[Android][Success]Customer can see suitability test (CSA). &Story &3\\
%     DA-14535 &[iOS][Success] Customer can see and accept result suitability test (CSA). &Story &2\\
%     DA-14822 &[Android][Success] Customer can see and accept result suitability test (CSA). &Story &2\\
%     DA-14536 &[iOS][Fail] Customer can see error dialog when get result suitability from legal-doc not available. &Story &1\\
%     DA-14823 &[Android][Fail] Customer can see error dialog when get result suitability from legal-doc not available. &Story &1\\
%     DA-14684 &[iOS][Fail] Customer can see error dialog when accept result suitability from legal-doc not available. &Story &1\\
%     DA-14840 &[Android][Fail] Customer can see error dialog when accept result suitability from legal-doc not available. &Story &1\\
%     DA-14807 &[Backend] Create order bond. &Story &5\\
%     DA-14527 &[iOS][Success] Customer can see review order page. &Story &3\\
%     DA-14790 &[Android][Success] Customer can see review order page. &Story &3\\
%     DA-14792 &[Android][Fail] Customer can see error dialog when create order not available. &Story &1\\
%     DA-14681 &[iOS][Success] Customer can see booking history &Story &2\\
%     DA-14871 &[Android][Success] Customer can see booking history &Story &2\\
%     DA-14528 &[iOS][Fail] Customer can see error dialog when create order not available. &Story &1\\
%     DA-14682 &[iOS][Fail] Customer see error when cannot get booking history &Story &1\\
%     DA-14873 &[Android][Fail] Customer see error when cannot get booking history &Story &1\\
%     DA-15333 &[Android][Success] Initial Mini app payment processor &Task &1\\
%     DA-15332 &[iOS][Success] Initial Mini app payment processor &Task &1\\
%     DA-15536 &[iOS][Success] Customer can see passcode for confirm booking bond &Story &1\\
%     DA-15560 &[iOS][Fail] Customer can't booking bond when customer never have customer type. &Story &0\\
% 	\hline
% \end{tabularx}

% จากตารางที่ \ref{tableOfStoryBookingSP22}  นี้จะเป็น Issue ของ Treasury - Booking ที่เกี่ยวกับส่วนของการจองซื้อหุ้นกู้โดยจะประกอบไปด้วยสองส่วนหลัก ๆ คือ ส่วนการจองซื้อ 
% และแบบประเมินความเสี่ยงของผู้ลงทุน ซึ่งทีม Treasury ได้รับผิดชอบในส่วนแบบประเมินความเสี่ยงของผู้ลงทุน และการสร้างคำสั่งซื้อ โดยลูกค้าจะทำแบบประเมินความเสี่ยงของผู้ลงทุนเมื่ออายุการประเมินความเสี่ยงของลูกค้าหมดอายุ

% \begin{tabularx}{\linewidth}{|c|X|c|c|}
% 	\caption{รายละเอียดของ Issue ของ Treasury - Booking (Sprint 23)}\label{tableOfStoryBookingSP23} \\
% 	\hline
%     \multicolumn{1}{|c|}{\textbf{Issue ID}}	&	\multicolumn{1}{c|}{\textbf{Issue}} &	\multicolumn{1}{c|}{\textbf{Issue Type}}  &	\multicolumn{1}{c|}{\textbf{Story Point}} \\
% 	\hline
% 	\endfirsthead
% 	\caption* {\textbf{ตารางที่ \ref{tableOfStoryBookingSP23} (ต่อ)} รายละเอียดของ Issue ของ Treasury - Booking (Sprint 23)} \\
% 	\hline
% 	\multicolumn{1}{|c|}{\textbf{Issue ID}}	&	\multicolumn{1}{c|}{\textbf{Issue}} &	\multicolumn{1}{c|}{\textbf{Issue Type}}  &	\multicolumn{1}{c|}{\textbf{Story Point}} \\
% 	\hline
% 	\endhead
% 	\hline
% 	\endfoot
% 	DA-15530 &[iOS][Fail] Customer input passcode wrong. &Story &1\\
% 	DA-15558 &[iOS][Fail] Customer can't confirm booking bond when insufficient amount&Story &1\\
%     DA-15552 &[iOS][Fail] Customer input passcode exceed pin attempt.&Story &2\\
%     DA-15541 &[iOS][Fail] Customer can't confirm booking bond when transaction limit.&Story &1\\
%     DA-15557 &[iOS][Fail][Confirm Step] Customer can't confirm booking bond when exceed quota limit. &Story &1\\
%     DA-15555 &[iOS][Fail] Customer input passcode and show error dialog when service not available &Story &1\\
%     DA-15531 &[iOS][Success] Customer can see slip after confirm booking bond &Story &3\\
%     DA-15533 &[iOS][Success] Save slip image after comfirm booking's bond is success &Story &1\\
%     DA-15534 &[iOS][Fail] Customer can't allow permission when save slip. &Story &1\\
%     DA-15535 &[Web Operation] Admin can see transaction history. &Story &1\\
%     DA-15693 &[iOS][Success] Customer can booking bond when Customer type match Product Type. &Story &2\\
%     DA-15537 &[iOS][Fail] Customer can't booking bond when Customer type lower than Product Type. &Story &1\\
%     DA-15556 &[API] Handle error create booking &Story &2\\
%     DA-15559 &[API] Handle error confirm booking. &Story &2\\
% 	\hline
% \end{tabularx}

% จากตารางที่ \ref{tableOfStoryBookingSP23}  นี้จะเป็น Issue ของ Treasury - Booking ที่เกี่ยวกับส่วนของการยืนยันคำสั่งซืิ้อหุ้นกู้ และการแจ้งเตือนผ่านอีเมล 
% โดยการยืนยันคำสั่งซื้อหุ้นกู้นั้นจะต้องมีการใส่รหัสผ่านที่ลูกค้าได้ตั้งไว้เพื่อระบุตัวตนว่าคือลูกค้าคนนั้นจริง ๆ และเมื่อยืนยันคำสั่งซื้อลูกค้าจะต้องเห็นหน้าใบเสร็จ 
% และสามารถบันทึกรูปภาพของใบเสร็จสั่งซื้อลงอัลบั้มภาพของโทรศัพท์มือถือได้อัตโนมัติ

% โดยที่ Issue แต่ละอันนั้นจะมี Acceptance Criteria ที่แตกต่างกันตามที่หัวหน้านักพัฒนาได้คุยกับ Product Owner ไว้แต่ละ Issue มีเป้าหมายอะไร
% และแต่ละทีมนั้นจะมีนิยามของคำเสร็จที่แตกต่างกันโดยที่ทีม Treasury นั้นมีนิยามว่านักพัฒนาทุกคนไม่ว่าจะพัฒนาในส่วนไหนต้องสร้างชุดทดสอบให้สำเร็จด้วย
% ตามส่วนที่ตนเองนั้นได้พัฒนาถ้าเป็นในส่วนของ Mobile Application ต้องทำ UI automated test ให้สำเร็จ และถ้าพัฒนาฝั่ง API ต้องทำ Unit test
% ให้เสร็จสิ้น

% \subsection{Sprint Planning}
% ในส่วนนี้จะเป็นส่วนที่นักพัฒนาทุกคนต้องมาช่วยกันวางแผน และสร้าง Sequence Diagram เพื่อที่ฝั่ง Frontend และ Backend
% จะได้ทำความเข้าใจว่าจะเกิดอะไรขึ้นบ้างเมื่อแอปพลิเคชันอยู่ในหน้าต่างนั้น ๆ โดยที่จะพูดถึงเรื่องการร้องขอข้อมูลจาก Frontend
% ไปยัง Backend หรือ Backend ร้องขอข้อมูลกันเองโดยทางทีม Treasury แบ่งได้เป็น 2 ส่วนหลักคือ

% \setcounter{secnumdepth}{5} 
% \subsubsection{Management Part}
% ส่วน Management จะเป็นส่วนจัดการข้อมูลของหุ้นกู้ทั้งหมดไม่ว่าจะเป็นการเพิ่ม ลบ และแก้ไขโดยมี Sequence Diagram ดังนี้

% \paragraph{เพิ่มข้อมูลหุ้นกู้}
% \begin{figure}[H]
%     \centering
%     \includegraphics[width=0.5\textwidth]{bond-information-add}
%     \caption{Sequence Diagram ของขั้นตอนการเพิ่มข้อมูลหุ้นกู้}\label{bond-information-add}
% \end{figure}
% การทำงานคือเมื่อ Web Operation Tools เพิ่มหุ้นกู้เข้าไปยัง Backend จะมีการตรวจสอบก่อนว่าหุ้นกู้นั้นมีอยู่ในระบบหรือยัง
% ถ้ามีแล้วจะไปตรวจสอบ Flag ที่สร้างไว้ตัวหนึ่งว่าหุ้นกู้นั้นมีสถานะถูกลบหรือเปล่า ถ้าถูกลบจะอัพเดทหุ้นตัวเดิมแล้วเปลี่ยนสถานะเป็นไม่ถูกลบ
% แต่ถ้ายังไม่ถูกลบจะเพิ่มล้มเหลว และถ้ายังไม่มีหุ้นกู้ก็จะสามารถเพิ่มเข้าไปในระบบได้ทันที

% \paragraph{แก้ไขข้อมูลหุ้นกู้}
% \begin{figure}[H]
%     \centering
%     \includegraphics[width=0.5\textwidth]{bond-information-update}
%     \caption{Sequence Diagram ของขั้นตอนการแก้ไขข้อมูลหุ้นกู้}\label{bond-information-update}
% \end{figure}
% การทำงานคือเมื่อ Web Operation Tools ต้องการแก้ไขข้อมูลขั้นตอนแรกต้องมีการตรวจสอบก่อนว่าหุ้นกู้นี้มีอยู่จริงหรือเปล่า
% ถ้ามีอยู่ก็สามารถแก้ไขได้ตามปกติ แต่ถ้าไม่มีก็จะทำการแก้ไขล้มเหลว

% \paragraph{ลบข้อมูลหุ้นกู้}
% \begin{figure}[H]
%     \centering
%     \includegraphics[width=0.5\textwidth]{bond-information-delete}
%     \caption{Sequence Diagram ของขั้นตอนการลบข้อมูลหุ้นกู้}\label{bond-information-delete}
% \end{figure}
% การทำงานคือเมื่อ Web Operation Tools เต้องการลบข้อมูลขั้นตอนแรกต้องมีการตรวจสอบก่อนว่าหุ้นกู้นี้มีอยู่จริงหรือเปล่า
% ถ้ามีอยู่ก็สามารถลบได้ตามปกติ แต่ถ้าไม่มีก็จะทำการลบล้มเหลว

% \paragraph{เรียกข้อมูลหุ้นกู้}
% \begin{figure}[H]
%     \centering
%     \includegraphics[width=0.5\textwidth]{bond-information-get}
%     \caption{Sequence Diagram ของขั้นตอนการเรียกข้อมูลหุ้นกู้}\label{bond-information-get}
% \end{figure}
% การทำงานคือเมื่อ Web Operation Tools ต้องการเรียกดูข้อมูลขั้นตอนแรกต้องมีการตรวจสอบก่อนว่าหุ้นกู้นี้มีอยู่จริงหรือเปล่า
% ถ้ามีอยู่ก็สามารถเรียกดูได้ตามปกติ แต่ถ้าไม่มีก็จะทำการเรียกดูล้มเหลว

% \newpage
% \subsubsection{Main Part}
% โดยในส่วนนี้เป็นส่วนหลักในการจองซื้อหุ้นกู้ซึ่งแบ่งออกได้ดังนี้

% \paragraph{สร้างคำสั่งการจองซื้อหุ้นกู้}
% \begin{figure}[H]
%     \centering
%     \includegraphics[width=0.5\textwidth]{booking-create-order}
%     \caption{Sequence Diagram ของขั้นตอนสร้างคำสั่งการจองซื้อหุ้นกู้}\label{booking-create-order}
% \end{figure}
% การสร้างคำสั่งการจองซื้อนั้นเป็นขั้นตอนที่มี Logic มากมายที่ใช้เพราะว่าการที่จะสร้างคำสั่งการจองซื้อได้ต้องผ่านส่วนต่าง ๆ ของแอปพลิเคชันมาทั้งหมดตามที่กำหนดไว้
% จึงต้องตรวจสอบสถานะในการผ่านหน้าต่าง ๆ มา อีกทั้งยังมีการตรวจสอบข้อมูลทั้งหมดของลูกค้าว่าใช่ลูกค้าจริง ๆ หรือเปล่าที่เป็นผู้กดจองซื่้อ และสุดท้ายก็ตรวจสอบ
% ว่าหุ้นกู้ยังสามารถจองซื้อได้อยู่หรือไม่

% \paragraph{ยืนยันการจองซื้อหุ้นกู้}
% \begin{figure}[H]
%     \centering
%     \includegraphics[width=0.5\textwidth]{booking-confirm-order}
%     \caption{Sequence Diagram ของขั้นตอนยืนยันการจองซื้อหุ้นกู้}\label{booking-confirm-order}
% \end{figure}
% ในขั้นตอนนี้ก่อนที่จะมีการยืนยันได้นั้นต้องผ่านการสร้างคำสั่งการจองซื้อมาก่อนจึงจะสามารถกดยืนยันได้ และจะมีการตรวจสอบยอดเงินคงเหลือในบัญชี
% จำนวนหุ้นกู้ที่สามารถซื้อได้ต่อวัน และ จำนวนหุ้นกู้ที่เหลือเพียงพอต่อการจองซื้อหุ้นกู้หรือไม่

% \subsection{Develop And Test}
% เป็นช่วงที่นักพัฒนาทุกคนจะพัฒนาในส่วนต่าง ๆ ในแอปพลิเคชันตาม Issue ใน Scrum Active Sprint Board ดังรูปที่ \ref{scrum-active-sprint-board} 
% \begin{figure}[H]
%     \centering
%     \includegraphics[width=1\textwidth]{scrum-active-sprint-board}
%     \caption{หน้าการใช้งาน Scrum Active Sprint Board}\label{scrum-active-sprint-board}
% \end{figure}
% โดยจะมีสถานะต่าง ๆ ของ Issue ดังนี้
% \subsubsection{To Do}
% สถานะนี้เป็น Issue ที่มาจาก Product Backlog เป็น Issue ทั้งหมดที่ต้องทำให้เสร็จภายใน Sprint นี้
% \subsubsection{In Progress}
% สถานะนี้เป็นสถานะที่ Issue กำลังถูกทำอยู่โดยถูกมอบหมายให้กับนักพัฒนาแต่ละคนรับผิดชอบ ซึ่งนักพัฒนาหนึ่งคนสามารถถือ Issue ได้เพียงแค่คนละ 1 Issue เพราะว่า
% ในกรณีที่นักพัฒนาถือ Issue ไว้พร้อมกัน 2 Issue แล้วนักพัฒนาทำอีกส่วนหนึ่งยังไม่เสร็จจะเป็นการ Block การทำงานของพัฒนาคนอื่นให้ช้ากว่าเดิม
% \subsubsection{Ready To Test}
% สถานะนี้เป็นสถานะที่ Issue ถูกพัฒนาเสร็จแล้วตาม Acceptance Criteria และส่งมอบต่อให้ Quality Assurance เพื่อทดสอบตาม Acceptance Criteria ที่ถูกตั้งไว้
% \subsubsection{Testing}
% สถานะนี้เป็นสถานะที่ Issue ถูกทดสอบโดย Quality Assurance หากมีกรณีไหนที่นักพัฒนาทำไม่ถูกต้องตาม Acceptance Criteria จะถูก Quality Assurance ลากไป Issue กลับไปยัง To Do เพื่อให้นักพัฒนาแก้ไข้ให้ถูกต้อง
% \subsubsection{Done}
% สถานะนี้เป็นสถานะที่ Issue ถูกพัฒนาเสร็จสิ้น ถูกต้องตาม Acceptance Criteria ทั้งหมด และเสร็จตามทั้งหมดตามนิยามของคำว่าเสร็จของทีมซึ่งคือการที่นักพัฒนาทุกคนต้องเขียนชุดทดสอบในส่วนที่ตนเองนั้นได้พัฒนาให้เสร็จสิ้น

% จากข้างต้นที่ว่าแต่ละ Issue นั้นจะ Acceptance Criteria ที่แตกต่างกันไปตามที่ Product Owner ได้กำหนดไว้ซึ่งจะมีการเขียนรายละเอียดเอาไว้ในคำอธิบายของ Issue ดังตัวอย่างจากรูปที่ \ref{acceptance-criteria}
% \begin{figure}[H]
%     \centering
%     \includegraphics[width=1\textwidth]{acceptance-criteria}
%     \caption{ตัวอย่างของการกำหนด Acceptance Criteria}\label{acceptance-criteria}
% \end{figure}

% หลังจากที่นักพัฒนาได้พัฒนา Issue ที่ตนเองได้รับผิดชอบเสร็จสิ้นแล้วในสถานนะของ Ready to Test นักพัฒนาต้องทำการ Push โค้ดส่วนที่พัฒนาเสร็จขึ้นไปยัง Git 
% เพื่อที่จะให้ Jenkins ที่เป็น CI/CD ของ Digital Banking นั้น Build ส่วนพัฒนาให้อยู่ในรูปของ Dependencies ที่ Super App สามารถเรียกไปใช้งานได้
% เพื่อที่จะได้ให้ Quality Assurance ได้ทดสอบในสภาพแวดล้อมของ SIT

% \subsection{Daily Stand-up}
% เป็นกิจกรรมที่จะมีการมาอัพเดทงานและปัญหาที่เจอในวันก่อน ๆ และบอกสิ่งที่จะทำในวันนี้ซึ่งจะทำในทุก ๆ เช้าของวันทำงานโดยถ้ามีปัญหาเกิดขึ้น
% ทีมจะช่วยกันหาทางแก้ปัญหาเพื่อให้นักพัฒนาคนนั้นสามารถทำงานต่อได้ เช่น 
% เมื่อเกิดปัญหาที่นักพัฒนาคนนั้นมีส่วนที่ต้องรอทีมอื่นแก้ไขในส่วนที่ทีมอื่นทำผิดพลาดอยู่ ทำให้นักพัฒนาในทีมไม่สามารถทำงานได้เพราะต้องรอส่วนที่ผิดพลาดนั้นอยู่
% ซึ่งในกรณีนี้ทุกคนในทีมก็จะหาทางมาแก้ไขปัญหาให้ได้ เพื่อที่งานในแต่ละวันนั้นจะได้ทำให้เสร็จสิ้น

% \subsection{Product Backlog Refinement}
% เป็นวันแรกของสัปดาห์ที่สองที่หัวหน้านักพัฒนาจะไปคุยกับผู้จัดการนักพัฒนา และหัวหน้านักพัฒนาทีมอื่น ๆ ว่างานที่เหลือใน Sprint นี้ยังเหมาะสมกับเวลาที่เหลือหรือเปล่า ถ้าในกรณีที่มีงานมากเกินไปหัวหน้านักพัฒนาจะเอา Issue ออกจาก Scrum Active Sprint Board
% แต่กลับกันในกรณีที่งานน้อยเกินไปกว่าเวลาที่เหลือทีมอาจจะได้รับงานเข้ามาทำเพิ่มเติม ซึ่งขึ้นอยู่กับดุลพินิจของหัวหน้านักพัฒนาว่าจะรับงานเข้ามาหรือว่าจะเอางานออก

% \subsection{Sprint Review}
% เป็นวันที่ 2 ของสัปดาห์ที่ 3 ที่นักพัฒนามาสาธิตงานที่ทำใน Sprint นี้หลังจากที่พัฒนา โดยจะสาธิตตาม Issue แต่ละ Issue ที่จะมี Acceptance Criteria 
% เป็นตัวชี้วัดว่าระบบสมบูรณ์หรือไม่ โดยผู้ที่ฟังการสาธิตนี้คือ Product Owner และ Project Manager ที่เป็นผู้ไปรับความต้องการทางธุรกิจจากหน่วยธุรกิจมา
% หลังจากที่มีการสาธิตจากนักพัฒนาเสร็จสิ้น Product Owner และ Project Manager จะต้องไปสาธิตกับทีม Business ทั้งหมดที่ดูแลส่วนต่าง ๆ ในแอปพลิเคชัน
% ว่ามีกระบวนการการทำงานอย่างไร

% \subsection{Release Code Day}
% เป็นวันที่นักพัฒนาจะทำการส่งมอบแอปพลิเคชันที่ทำใน Sprint นี่ขึ้นสู่สภาพแวดล้อมของ 
% User Acceptance Testing โดยจะมีการส่งมอบเป็นลำดับตามความสำคัญของส่วนต่าง ๆ ในแอปพลิเคชัน

% \subsection{User Acceptance Testing}
% เป็นช่วงที่ Quality Assurance หรือ QA จะทำการทดสอบแอปพลิเคชันโดยการจำลองเป็นลูกค้าแล้วหาจุดผิดพลาดเพื่อให้นักพัฒนาแก้ไขโดยจะเริ่มตั้งแต่วันที่ 3 ของสัปดาห์ที่สามจนถึงวันสุดท้ายของสัปดาห์ที่สาม
% ถ้ามีการแก้ไขในช่วงนี้จะไม่นับการแก้ไขนั้นเป็นงานใน Sprint แต่จะนับเป็นการ Hotfix

% \subsection{PI Planning}
% เป็นวันใดวันหนึ่งก็ได้ใน Sprint ที่หัวหน้านักพัฒนาจะคุยกับ Product Owner ถึง Product Backlog ใน Sprint 
% ว่าจะต้องทำอะไรบ้างความต่้องการทางธุรกิจที่ไปเจรจาและรวบรวมมาจากทีม Business
        
% \subsection{Sprint Retrospective}
% เป็นวันสุดท้ายของสัปดาห์ที่สามที่ทุก ๆ คนในทีมไม่ว่าจะเป็น นักพัฒนา, หัวหน้านักพัฒนา, Quality Assurance, Project Manager 
% และ Product Owner จะมาพูดถึงเรื่องราวต่าง ๆ ที่เกิดขึ้นใน Sprint ที่ผ่านมาไม่ว่าจะเป็นเรื่องงาน 
% หรือเรื่องความเป็นอยู่ของคนในทีมโดยทาง Digital Banking จะเลือกใช้รูปแบบการ 
% Retrospective เป็น Good, Bad, Try และ Next Action ดังรูปที่
% โดยสามารถอธิบายรายละเอียดได้ดังนี้
% \subsubsection{Good}
% จะเป็นส่วนที่คนในทีมบอกว่ามีอะไรบ้างใน Sprint นี้ที่เป็นเรื่องดีไม่ว่าจะเป็นการทำงานเสร็จทันเวลา การมีทีมเวิร์คที่ดี หรือการได้เรียนรู้สิ่งใหม่ ๆ
% เป็นต้น
% \subsubsection{Bad}
% จะเป็นส่วนที่คนในทีมบอกว่ามีอะไรบ้างใน Sprint นี้ที่เป็นเรื่องแย่ไม่ว่าจะเป็นการทำงานช้า การรองานกัน การสื่อสารที่ไม่ดี คอมพิวเตอร์ไม่ดี หรือ
% ทำงานไม่เข้ากับคนทีมเป็นต้น
% \subsubsection{Try}
% จะเป็นส่วนที่คนในทีมบอกว่าใน Sprint หน้ามีอะไรที่ควรพยายามหรือพัฒนาให้มากขึ้น ไม่ว่าจะเป็นวิธีการทำงาน การสื่อสารที่ไม่ดี หรืออยากให้ทำงานให้เร็วกว่านี้เป็นต้น
% \subsubsection{Next Action}
% จะเป็นส่วนที่ Facilitator จะให้เปิดให้โหวตให้คนในทีมเลือกการ์ดที่ควรทำใน Sprint ถัดไปโดยจะเลือกจากการ์ดที่มีคนเลือกมากที่สุดสองใบ


    \chapter{ปัญหาและข้อเสนอแนะ}
\thispagestyle{empty}
\label{chapter:result}

\section{ปัญหาด้านสถานประกอบการ}
    \subsection{ปัญหา}
        ปัญหาในการสื่อสารภายในองค์กรทำให้ผู้ได้รับข่าวสารอาจได้รับข่าวสารที่คลาดเคลื่อน
    \subsection{ข้อเสนอแนะ}
        อยากให้มีการสื่อสารภายในองค์กรอย่างชัดเจนทำให้สามารถเข้าใจตรงกัน

\section{ปัญหาด้านสถาบัน}
    \subsection{ปัญหา}
        ปัญหาการในการให้ความเข้าใจกับสถานประกอบการว่าโครงการสหกิจต้องเป็นงานที่มีปริมาณขนาดไหนทำให้เตรียมไม่ถูก
    \subsection{ข้อเสนอแนะ}
        อยากให้ทางสถาบันออกเป็นรายการตรวจสอบไว้เพื่อให้ทางสถานประกอบการได้ไว้เพื่อใช้ประกอบการตัดสินใจรับนักศึกษาเข้าโครงการสหกิจศีกษาว่ามีโครงงานตามที่โครงการสหกิจต้องการหรือไม่
\section{ปัญหาด้านตัวนักศึกษา}
    \subsection{ปัญหา}
        ปัญหาของนักศึกษาคือการสื่อสารกับพี่ที่เป็นที่ปรึกษาเนื่องจากไม่ทราบจะเข้าหายังไงให้ถูกจังหวะ
    \subsection{ข้อเสนอแนะ}
        นักศึกษาควรสังเกตการทำงานของพี่ที่จะเข้าไปสื่อสารด้วยให้มากกว่านี้เพื่อหาเวลาในการเข้าไปสื่อสารอย่างไม่ผิดกาละเทศะ

    % \chapter{บทสรุป}
\thispagestyle{empty}
\label{chapter:conclusion}

% \section{วิเคราะห์ผลการปฏิบัติงาน}
% จากการที่นักศึกษาปฏิบัติงานสหกิจศึกษา ณ ธนาคาร ซีไอเอ็มบีไทย จำกัด (มหาชน) ตั้งแต่วันที่ \StartDWork จนถึง \EndDWork 
% รวมเป็นเวลา 5 เดือน 17 วัน ซึี่งนักศึกษาได้มีส่วนร่วมในการช่วยพัฒนาโมบายล์แอปพลิเคชันบนระบบปฏิบัติการ Android และ iOS
% ในส่วนของการจองซื้อหุ้นกู้นั้น ทำให้ข้าพเจ้าได้รับประสบการณ์หลากหลายรูปแบบที่สอนให้ข้าพเจ้าได้รู้จักการทำงานแบบทีม ไม่ว่าจะเป็นการ
% คุยงาน วางแผน กำหนดขอบเขตการพัฒนา ข้อผิดพลาดต่าง ๆ ที่เกิดขึ้นระหว่างพัฒนา และการแก้ปัญหาที่เกิดขึ้นจริงระหว่างพัฒนา 
% อีกทั้งยังได้เรียนรู้เทคโนโลยีต่าง ๆ ที่ช่วยให้เราพัฒนาซอฟต์แวร์ได้มีประสิทธิภาพมากขึ้น นอกจากนี้ยังได้ทักษะการสื่อสารกับคนในทีมทั้งรูปแบบการพูด
% และเรียนรู้การใช้ภาษาอังกฤษในการทำงาน สุดท้ายการที่ได้มาร่วมงาน ณ ที่แห่งนี้ทำให้ข้าพเจ้า ได้เพิ่มทักษะในการวิเคราะห์ เพิ่มความมีระเบียบวินัย
% รู้จักความรับผิดชอบในหน้าที่ของตนเองมากขึ้น ซึ่งทั้งหมดที่กล่าวมานั้นจะหล่อหลอมให้ข้าพเจ้าสามารถนำประสบการณ์ และความรู้นี้ไปใช้ในการทำงานจริงได้ในอนาคต

% \section{ประโยชน์ที่ได้รับจากการปฏิบัติงาน}
% \subsection{ประโยชน์ต่อตนเอง} 
% พัฒนาความรู้ความสามารถจากการนำเรื่องที่ศึกษาในคณะเทคโนโลยีสารสนเทศมาต่อยอดในสถานประกอบการจริง
% อีกทั้งยังสามารถนำเรื่องต่าง ๆ ที่เรียนรู้มาใช้ต่อในอนาคตได้อีกด้วย

% \subsection{ประโยชน์ต่อสถานประกอบการ}
% สถานประกอบการมีช่องทางการจองซื้อหุ้นกู้เพิ่มจากเดิมที่ต้องทำการจองซื้อผ่านสาขาอย่างเดียว ทำให้ลูกค้ามีความสะดวกและพึงพอใจมากขึ้น
% อีกทั้งยังเป็นการเพิ่มรายได้ให้แก่สถานประกอบการอีกด้วย

% \subsection{ประโยชนต่อมหาวิทยาลัย}
% ทำให้สถานประกอบการมองว่ามีนักศึกษาที่สามารถทำงานได้จริงในปัจจุบันก่อนศึกษาจบ 
% เพิ่มความเชื่อให้แก่สถาณประกอบการต่าง ๆ ว่านักศึกษาที่มาจากสถาบันนี้มีคุณภาพ

% \section{วิเคราะห์จุดเด่น จุดด้อย โอกาส อปสรรค (SWOT Analysis)}
% \subsection{จุดเด่น}  
% เป็นคนที่ใช้เวลาในการเรียนรู้น้อย และเข้าใจสิ่งต่าง ๆ ได้ง่าย สามารถเริ่มงานได้ไว ตั้งใจทำงานอย่างเต็มที่
% มีความมุ่งมั่นสูง มีความคิดในการแก้ปัญหาเมื่อทีมเจอปัญหา

% \subsection{จุดด้อย}
% เป็นคนที่ติดการถามก่อนทำจริงซึ่งในโลกแห่งความเป็นจริงแล้วไม่มีใครสอนเราได้ดีกว่าการเรียนรู้ด้วยตนเอง
% เป็นคนที่มีความคิดที่ไม่ถี่ถ้วน คิดไม่รอบคอบ ชอบทำอะไรเกินความต้องการเมื่อมีคำสั่งมา
% ปัญหา

% \subsection{โอกาส}
% ได้มีเรียนรู้สิ่งใหม่ ๆ ทั้งเรื่องของความรู้ สังคม และการใช้ชีวิตในการทำงานจริง
% ความรู้ได้ทั้งการเรียนรู้การพัฒนา Mobile Application ที่ตัวนักศึกษานั้นไม่เคยได้ทำมาก่อน
% ทั้ง iOS และ Android นักศึกษาได้มีโอกาสสื่อสารภาษาอังกฤษบ่อยครั้งเนื่องจากมีเพื่อนร่วมที่เป็นคนต่างชาติ

% \subsection{อุปสรรค}
% นักพัฒนาในทีมน้อยเกินไปทำงานหนักไปในบาง Sprint ที่มี Issue เยอะ ๆ ทำให้ต้องทำงานร่วงเวลาในบางครั้ง
% และยังมีการที่มี Dependencies กับทีมอื่นที่ Block การพัฒนาของทีมจึงทำให้เกิดการพัฒนาที่ช้าในบางส่วน

% \section{ปัญหา และข้อเสนอแนะ}
% \subsection{ด้านมหาวิทยาลัย}
% ระบบจัดการนักศึกษาสหกิจศึกษาไม่สมบูรณ์ขาดความชัดเจนในหลาย ๆ ส่วนทำให้นักศึกษา และสถานประกอบการไม่เข้าใจว่า
% ใครจะเป็นคนประสานงาน อีกทั้งยังมีการประชาสัมพันธ์ที่ไม่ชัดเจน เอกสารต่าง ๆ ที่ไม่มีมาตรฐานที่ชัดเจน 

% \noindent \textbf{ข้อเสนอแนะ:} ควรปรับปรุงระบบนักศึกษาสหกิจศึกษาให้ดีกว่านี้ และอยากให้เน้นการปฏิบัติงาน
% ของนักศึกษามากกว่าการดูในผลงานที่เป็นชิ้นงาน เนื่องจากอาจจะไม่มีชิ้นงานที่เป็นกิจลักษณะ 
% อยากให้ดูว่านักศึกษาได้อะไรจากฝึกงานมากกว่า

% \subsection{ด้านตัวนักศึกษา}
% เป็นคนถามก่อนที่จะได้ลงมือปฏิบัติงานจริง เป็นคนตอบตกลงโดยไม่ประเมินก่อนว่าตนเองนั้นทำได้หรือทำไม่ได้
% เป็นคนพูดเร็ว คิดแล้วทำอะไรไวเกินไป ทำให้การสื่อสารบางส่วนกับคนในทีมไม่เข้าใจ หรือไม่ตรงกัน

% \noindent \textbf{ข้อเสนอแนะ:} ควรปรับปรุงในเรื่องของความการลองลงมือทำก่อนมีปัญหาแล้วลองแก้ไขเอง 
% ถ้าแก้ปัญหาแล้วทำไมได้ค่อยถาม ต้องประเมินงานก่อนที่จะตกลงในทุก ๆ ด้านว่าตนเองทำได้หรือไม่ และต้องเป็นคนที่พูดช้ากว่านี้
% เพื่อที่การสื่อสารจะได้ดีกว่านี้

% \subsection{ด้านสถานประกอบการ}
% เนื่องจากสถานประกอบการได้รับนักศึกษาสหกิจศึกษาเป็นครั้งแรก (ในแผนก Digital Banking) ทำให้ยังไม่ค่อยรู้เรื่องวิธีการ
% จัดการกับนักศึกษาและโปรเจคทำให้เรื่องการหาโปรเจคให้นักศึกษานั้นล่าช้า และ Scope ของโปรเจคยังไม่ชัดเจน และมีจำนวน

% \noindent \textbf{ข้อเสนอแนะ:} ควรมีโปรเจคที่ Scope ชัดเจนกว่านี้ที่ให้นักศึกษาได้ทำ

    
    \clearpage
    \addcontentsline{toc}{chapter}{บรรณานุกรม}
    \bibliographystyle{IEEEtran}
    \bibliography{reference}
    
    \startappendix
    \clearpage 
\thispagestyle{empty}
\begin{center}
	\vspace*{\stretch{1}}
	\LARGE{\textbf{ภาคผนวก ก}}
	\vspace*{\stretch{1}}
\end{center}

	\chapter{บันทึกเวลาการปฏิบัติงาน}

	\begin{tabularx}{\linewidth}{|c|c|c|c|}
		\caption{รายงานบันทึกเวลาปฏิบัติงานประจำเดือน มิถุนายน}\label{timeSheetJun} \\
		\hline
		\multicolumn{1}{|c|}{\textbf{วัน/เดือน/ปี}}	&	\multicolumn{1}{c|}{\textbf{เวลาเข้าทำงาน}} &	\multicolumn{1}{c|}{\textbf{เวลากลับ}} &	\multicolumn{1}{c|}{\textbf{หมายเหตุ}} \\
		\hline
		\endfirsthead
		\caption* {\textbf{ตารางที่ \ref{timeSheetJun} (ต่อ)} รายงานบันทึกเวลาปฏิบัติงานประจำเดือน มิถุนายน} \\
		\hline
		\multicolumn{1}{|c|}{\textbf{วัน/เดือน/ปี}}	&	\multicolumn{1}{c|}{\textbf{เวลาเข้าทำงาน}} &	\multicolumn{1}{c|}{\textbf{เวลากลับ}} &	\multicolumn{1}{c|}{\textbf{หมายเหตุ}} \\
		\hline
		\endhead
		\hline
		\endfoot
		1/มิ.ย./2563 &09.00 & 19.02 & \ \\
		2/มิ.ย./2563 &07.43 & 18.45 & \ \\
		4/มิ.ย./2563 &09.00 & 18.00 & \ \\
		5/มิ.ย./2563 &07.31 & 18.12 & \ \\
		8/มิ.ย./2563 &07.22 & 18.51 & \ \\
		9/มิ.ย./2563 &07.38 & 18.16 & \ \\
		10/มิ.ย./2563 &09.00 & 18.00 & \ \\
		11/มิ.ย./2563 &08.30 & 18.48 & \ \\
		12/มิ.ย./2563 &08.21 & 19.13 & \ \\
		15/มิ.ย./2563 &08.53 & 18.21 & \ \\
		16/มิ.ย./2563 &08.42 & 18.13 & \ \\
		17/มิ.ย./2563 &08.46 & 18.13 & \ \\
		18/มิ.ย./2563 &08.52 & 18.12 & \ \\
		19/มิ.ย./2563 &- & - & ลาป่วย \\
		22/มิ.ย./2563 &08.42 & 18.17 & \ \\
		23/มิ.ย./2563 &08.26 & 18.07 & \ \\
		24/มิ.ย./2563 &08.39 & 18.50 & \ \\
		25/มิ.ย./2563 &08.28 & 18.07 & \ \\
		26/มิ.ย./2563 &08.29 & 18.05 & \ \\
		29/มิ.ย./2563 &08.49 & 18.06 & \ \\
		30/มิ.ย./2563 &07.42 & 18.09 & \ \\
		\hline
	\end{tabularx}

	\begin{tabularx}{\linewidth}{|c|c|c|c|}
		\caption{รายงานบันทึกเวลาปฏิบัติงานประจำเดือน กรกฎาคม}\label{timeSheetJul} \\
		\hline
		\multicolumn{1}{|c|}{\textbf{วัน/เดือน/ปี}}	&	\multicolumn{1}{c|}{\textbf{เวลาเข้าทำงาน}} &	\multicolumn{1}{c|}{\textbf{เวลากลับ}} &	\multicolumn{1}{c|}{\textbf{หมายเหตุ}} \\
		\hline
		\endfirsthead
		\caption* {\textbf{ตารางที่ \ref{timeSheetJul} (ต่อ)} รายงานบันทึกเวลาปฏิบัติงานประจำเดือน กรกฎาคม} \\
		\hline
		\multicolumn{1}{|c|}{\textbf{วัน/เดือน/ปี}}	&	\multicolumn{1}{c|}{\textbf{เวลาเข้าทำงาน}} &	\multicolumn{1}{c|}{\textbf{เวลากลับ}} &	\multicolumn{1}{c|}{\textbf{หมายเหตุ}} \\
		\hline
		\endhead
		\hline
		\endfoot
		1/ก.ค./2563 &07.39 & 18.30 & \ \\
		2/ก.ค./2563 &09.00 & 18.00 & \ \\
		3/ก.ค./2563 &09.00 & 18.00 & ลาป่วย \\
		4/ก.ค./2563 &08.45 & 18.00 & \ \\
		6/ก.ค./2563 &08.01 & 19.08 & \ \\
		7/ก.ค./2563 &07.38 & 18.50 & \ \\
		8/ก.ค./2563 &08.46 & 18.44 & \ \\
		9/ก.ค./2563 &07.33 & 18.55 & \ \\
		10/ก.ค./2563 &08.45 & 18.10 & \ \\
		13/ก.ค./2563 &08.50 & 18.00 & \ \\
		14/ก.ค./2563 &08.45 & 18.00 & \ \\
		15/ก.ค./2563 &08.50 & 18.00& \ \\
		16/ก.ค./2563 &08.55 & 18.00 & \ \\
		17/ก.ค./2563 &08.45 & 18.00 &\ \\
		20/ก.ค./2563 &09.00 & 18.00 &\ \\
		21/ก.ค./2563 &07.38 & 18.41 & \ \\
		22/ก.ค./2563 &09.00 & 20.00 & \ \\
		23/ก.ค./2563 &09.00 & 19.00 & \ \\
		24/ก.ค./2563 &08.32 & 18.07 & \ \\
		29/ก.ค./2563 &07.41 & 18.20 & \ \\
		30/ก.ค./2563 &- & - & ลาธุระส่วนตัว \\
		31/ก.ค./2563 &07.46 & 18.14 & \ \\
		\hline
	\end{tabularx}

	\begin{tabularx}{\linewidth}{|c|c|c|c|}
		\caption{รายงานบันทึกเวลาปฏิบัติงานประจำเดือน สิงหาคม}\label{timeSheetAug} \\
		\hline
		\multicolumn{1}{|c|}{\textbf{วัน/เดือน/ปี}}	&	\multicolumn{1}{c|}{\textbf{เวลาเข้าทำงาน}} &	\multicolumn{1}{c|}{\textbf{เวลากลับ}} &	\multicolumn{1}{c|}{\textbf{หมายเหตุ}} \\
		\hline
		\endfirsthead
		\caption* {\textbf{ตารางที่ \ref{timeSheetAug} (ต่อ)} รายงานบันทึกเวลาปฏิบัติงานประจำเดือน สิงหาคม} \\
		\hline
		\multicolumn{1}{|c|}{\textbf{วัน/เดือน/ปี}}	&	\multicolumn{1}{c|}{\textbf{เวลาเข้าทำงาน}} &	\multicolumn{1}{c|}{\textbf{เวลากลับ}} &	\multicolumn{1}{c|}{\textbf{หมายเหตุ}} \\
		\hline
		\endhead
		\hline
		\endfoot
		3/ส.ค./2563 &- & - & ลากลับสถาบัน \\
		4/ส.ค./2563 &- & - & ลากลับสถาบัน \\
		5/ส.ค./2563 &- & - & ลากลับสถาบัน \\
		6/ส.ค./2563 &- & - & ลากลับสถาบัน \\
		7/ส.ค./2563 &- & - & ลากลับสถาบัน \\
		10/ส.ค./2563 &07.40 & 18.03 & \ \\
		11/ส.ค./2563 &- & - & ลากลับสถาบัน \\
		12/ส.ค./2563 &- & - & ลากลับสถาบัน \\
		13/ส.ค./2563 &- & -& ลากลับสถาบัน \\
		14/ส.ค./2563 &- & - & ลากลับสถาบัน \\
		17/ส.ค./2563 &- & - & ลากลับสถาบัน \\
		18/ส.ค./2563 &09.00 & 18.00 & \ \\
		19/ส.ค./2563 &09.00 & 18.00 &\ \\
		20/ส.ค./2563 &09.00 & 18.00 & \ \\
		21/ส.ค./2563 &09.00 & 18.00 & \ \\
		24/ส.ค./2563 &09.00 & 18.00 & \ \\
		25/ส.ค./2563 &09.00 & 18.00 & \ \\
		26/ส.ค./2563 &09.00 & 18.00 & \ \\
		27/ส.ค./2563 &09.00 & 18.00 & \ \\
		28/ส.ค./2563 &07.14 & 18.05 & \ \\
		31/ส.ค./2563 &07.21 & 18.50 & \ \\
		\hline
	\end{tabularx}

	\begin{tabularx}{\linewidth}{|c|c|c|c|}
		\caption{รายงานบันทึกเวลาปฏิบัติงานประจำเดือน กันยายน}\label{timeSheetAugSep} \\
		\hline
		\multicolumn{1}{|c|}{\textbf{วัน/เดือน/ปี}}	&	\multicolumn{1}{c|}{\textbf{เวลาเข้าทำงาน}} &	\multicolumn{1}{c|}{\textbf{เวลากลับ}} &	\multicolumn{1}{c|}{\textbf{หมายเหตุ}} \\
		\hline
		\endfirsthead
		\caption* {\textbf{ตารางที่ \ref{timeSheetSep} (ต่อ)} รายงานบันทึกเวลาปฏิบัติงานประจำเดือน กันยายน} \\
		\hline
		\multicolumn{1}{|c|}{\textbf{วัน/เดือน/ปี}}	&	\multicolumn{1}{c|}{\textbf{เวลาเข้าทำงาน}} &	\multicolumn{1}{c|}{\textbf{เวลากลับ}} &	\multicolumn{1}{c|}{\textbf{หมายเหตุ}} \\
		\hline
		\endhead
		\hline
		\endfoot
		1/ก.ย./2563 &08.32 & 18.00 & \ \\
		2/ก.ย./2563 &07.28 & 18.00  & \ \\
		3/ก.ย./2563 &09.00 & 18.00 & \ \\
		8/ก.ย./2563 &08.32 & 18.07 & \ \\
		9/ก.ย./2563 &08.14 & 18.00  & \ \\
		10/ก.ย./2563 &07.43 & 18.07 & \ \\
		11/ก.ย./2563 &07.49 & 18.05 & \ \\
		14/ก.ย./2563 &06.47 & 18.00 & \ \\
		15/ก.ย./2563 &06.51 & 18.00 & \ \\
		16/ก.ย./2563 &07.49 & 18.12  &\ \\
		17/ก.ย./2563 &09.00 & 18.00 &\ \\
		18/ก.ย./2563 &07.03 & 18.12 & \ \\
		21/ก.ย./2563 &07.21 & 18.08 &\ \\
		22/ก.ย./2563 &06.57 & 18.10 & \ \\
		23/ก.ย./2563 &08.52 & 18.07 & \ \\
		24/ก.ย./2563 &08.48 & 18.06 & \ \\
		25/ก.ย./2563 &08.52 & 18.14 & \ \\
		28/ก.ย./2563 &07.24 & 18.13 & \ \\
		29/ก.ย./2563 &07.24 & 18.13 & \ \\
		30/ก.ย./2563 &07.05 & 18.08 & \ \\
		\hline
	\end{tabularx}

	\begin{tabularx}{\linewidth}{|c|c|c|c|}
		\caption{รายงานบันทึกเวลาปฏิบัติงานประจำเดือน ตุลาคม}\label{timeSheetAugSepOct} \\
		\hline
		\multicolumn{1}{|c|}{\textbf{วัน/เดือน/ปี}}	&	\multicolumn{1}{c|}{\textbf{เวลาเข้าทำงาน}} &	\multicolumn{1}{c|}{\textbf{เวลากลับ}} &	\multicolumn{1}{c|}{\textbf{หมายเหตุ}} \\
		\hline
		\endfirsthead
		\caption* {\textbf{ตารางที่ \ref{timeSheetSep} (ต่อ)} รายงานบันทึกเวลาปฏิบัติงานประจำเดือน ตุลาคม} \\
		\hline
		\multicolumn{1}{|c|}{\textbf{วัน/เดือน/ปี}}	&	\multicolumn{1}{c|}{\textbf{เวลาเข้าทำงาน}} &	\multicolumn{1}{c|}{\textbf{เวลากลับ}} &	\multicolumn{1}{c|}{\textbf{หมายเหตุ}} \\
		\hline
		\endhead
		\hline
		\endfoot
		1/ต.ค./2563 &07.40 & 18.00 & \ \\
		2/ต.ค./2563 &07.57 & 18.00  & \ \\
		3/ต.ค./2563 &08.40 & 18.18 & \ \\
		5/ต.ค./2563 &07.25 & 18.19 & \ \\
		5/ต.ค./2563 &08.14 & 18.00  & \ \\
		6/ต.ค./2563 &07.25 & 18.19 & \ \\
		7/ต.ค./2563 &07.00 & 18.04 & \ \\
		8/ต.ค./2563 &07.26 & 18.04 & \ \\
		9/ต.ค./2563 &08.47 & 18.05 & \ \\
		12/ต.ค./2563 &07.59 & 18.07  &\ \\
		14/ต.ค./2563 &- & - &ลาธุระส่วนตัว \\
		15/ต.ค./2563 &09.00 & 18.00 & \ \\
		16/ต.ค./2563 &08.56 & 18.27 &\ \\
		19/ต.ค./2563 &09.07 & 19.05 & \ \\
		20/ต.ค./2563 &07.24 & 18.11 & \ \\
		21/ต.ค./2563 &08.42 & 18.08 & \ \\
		22/ต.ค./2563 &08.55 & 18.20 & \ \\
		23/ต.ค./2563 &08.55 & 18.19 & \ \\
		26/ต.ค./2563 &08.52 & 18.24 & \ \\
		27/ต.ค./2563 &08.00 & 18.12 & \ \\
		28/ต.ค./2563 &07.03 & 18.05 & \ \\
		29/ต.ค./2563 &08.58 & 18.07 & \ \\
		30/ต.ค./2563 &09.17 & 18.18 & \ \\
		\hline
	\end{tabularx}

\clearpage 
\thispagestyle{empty}
\begin{center}
	\vspace*{\stretch{1}}
	\LARGE{\textbf{ภาคผนวก ข}}
	\vspace*{\stretch{1}}
\end{center}

\chapter{บันทึกรายงานการปฎิบัติงาน}

% \begin{tabularx}{\linewidth}{|c|c|}
% 	\caption{บันทึกรายงานการปฎิบัติงาน ตุลาคม}\label{timeSheetDetailJuly} \\
% 	\hline
% 	\multicolumn{1}{|c|}{\textbf{วัน/เดือน/ปี}}	&	\multicolumn{1}{c|}{\textbf{เวลาเข้าทำงาน}} &	\multicolumn{1}{c|}{\textbf{เวลากลับ}} &	\multicolumn{1}{c|}{\textbf{หมายเหตุ}} \\
% 	\hline
% 	\endfirsthead
% 	\caption* {\textbf{ตารางที่ \ref{timeSheetDetailJuly} (ต่อ)} รายงานบันทึกเวลาปฏิบัติงานประจำเดือน ตุลาคม} \\
% 	\hline
% 	\multicolumn{1}{|c|}{\textbf{วัน/เดือน/ปี}}	&	\multicolumn{1}{c|}{\textbf{เวลาเข้าทำงาน}} &	\multicolumn{1}{c|}{\textbf{เวลากลับ}} &	\multicolumn{1}{c|}{\textbf{หมายเหตุ}} \\
% 	\hline
% 	\endhead
% 	\hline
% 	\endfoot
% 	1/ต.ค./2563 &07.40 & 18.00 & \ \\
% 	2/ต.ค./2563 &07.57 & 18.00  & \ \\
% 	3/ต.ค./2563 &08.40 & 18.18 & \ \\
% 	5/ต.ค./2563 &07.25 & 18.19 & \ \\
% 	5/ต.ค./2563 &08.14 & 18.00  & \ \\
% 	6/ต.ค./2563 &07.25 & 18.19 & \ \\
% 	7/ต.ค./2563 &07.00 & 18.04 & \ \\
% 	8/ต.ค./2563 &07.26 & 18.04 & \ \\
% 	9/ต.ค./2563 &08.47 & 18.05 & \ \\
% 	12/ต.ค./2563 &07.59 & 18.07  &\ \\
% 	14/ต.ค./2563 &- & - &ลาธุระส่วนตัว \\
% 	15/ต.ค./2563 &09.00 & 18.00 & \ \\
% 	16/ต.ค./2563 &08.56 & 18.27 &\ \\
% 	19/ต.ค./2563 &09.07 & 19.05 & \ \\
% 	20/ต.ค./2563 &07.24 & 18.11 & \ \\
% 	21/ต.ค./2563 &08.42 & 18.08 & \ \\
% 	22/ต.ค./2563 &08.55 & 18.20 & \ \\
% 	23/ต.ค./2563 &08.55 & 18.19 & \ \\
% 	26/ต.ค./2563 &08.52 & 18.24 & \ \\
% 	27/ต.ค./2563 &08.00 & 18.12 & \ \\
% 	28/ต.ค./2563 &07.03 & 18.05 & \ \\
% 	29/ต.ค./2563 &08.58 & 18.07 & \ \\
% 	30/ต.ค./2563 &09.17 & 18.18 & \ \\
% 	\hline
% \end{tabularx}

% % \usepackage{longtable}


% \usepackage{longtable}


\begin{longtable}{|l|l|}
	\caption{บันทึกรายงานปฏิบัติงานประจำเดือน มิถุนายน}\label{timeSheetDetailJuly} \\
	\hline
	\textbf{วันที่} & \textbf{รายละเอียดการทำงาน}            \\                                         
	\hline
	\endfirsthead
	\caption* {\textbf{ตารางที่ \ref{timeSheetDetailJuly} (ต่อ)} บันทึกรายงานปฏิบัติงานประจำเดือน มิถุนายน} \\
	\textbf{วันที่} & \textbf{รายละเอียดการทำงาน}            \\                                         
	\hline
	\endhead
	\hline
	\endfoot
	\hline
	01/มิ.ย./2563   & ปฐมนิเทศ                                                                        \\ 
	\hline
	02/มิ.ย./2563   & ปฐมนิเทศ                                                                        \\ 
	\hline
	04/มิ.ย./2563   & DS : เพิ่มข้อมูลที่หลุดของ vendor ไปยัง QPO                                     \\ 
	\hline
	05/มิ.ย./2563   & DS : เพิ่มข้อมูลที่หลุดของ vendor ไปยัง QPO                                     \\ 
	\hline
	08/มิ.ย./2563   & DS : แก้ไข Fastlane ให้ลง Unitprice                                             \\ 
	\hline
	09/มิ.ย./2563   & DS : แก้ไขการเลื่อนเวลาคิวให้ได้วันว่างจริงๆ                                    \\ 
	\hline
	10/มิ.ย./2563   & DS : แก้ไขการเลื่อนเวลาคิวให้ได้วันว่างจริงๆ                                    \\ 
	\hline
	11/มิ.ย./2563   & DS: กดเปลี่ยนประเภทบริการ ระบบไม่จองคิวใหม่ให้                                  \\ 
	\hline
	12/มิ.ย./2563   & DS : แก้ให้การจองบริการลง disount กับ total amount และ cod ไม่เป็น 0            \\ 
	\hline
	15/มิ.ย./2563   & DS : แก้ให้การจองบริการลง disount กับ total amount และ cod ไม่เป็น 0            \\ 
	\hline
	16/มิ.ย./2563   & SMS : แก้ให้ sms สามารถ เพิ่มกลุ่มได้ทุกคน และเวลาดูชื่อต้อง sender bind        \\ 
	\hline
	17/มิ.ย./2563   & DS : ให้ dscenter สามารถเลือกสาขาที่จะจัดส่งได้                                 \\ 
	\hline
	18/มิ.ย./2563   & DS : แก้ให้วันที่เกิดการ callback และหาค่าใหม่                                  \\ 
	\hline
	19/มิ.ย./2563   & ลาไปมหาวิทยาลัยเพื่อนำเสนอผลงาน                                                 \\ 
	\hline
	22/มิ.ย./2563   & PHY : ให้แสดงวันที่และเวลาก่อน lock และหลัง lock data                           \\ 
	\hline
	23/มิ.ย./2563   & PHY : ให้คำณวน show\_vendor ลงไปใน pdf ด้วย                                     \\ 
	\hline
	24/มิ.ย./2563   & TMS : DEV : เพิ่ม log การทำงาน autoroute                                        \\ 
	\hline
	25/มิ.ย./2563   & STU : เพิ่ม EMPNO\_FULL ใน Add token และให้ get user จาก EMPNO\_FULL แทน        \\ 
	\hline
	26/มิ.ย./2563   & STU : เพิ่ม EMPNO\_FULL ใน Add token และให้ get user จาก EMPNO\_FULL แทน        \\ 
	\hline
	29/มิ.ย./2563   & DEV : สร้างส่วนของ Home Decor และแก้ส่วน Dni ใหม่ใช้ด้วยกันโดยแยก channel H, D  \\ 
	\hline
	30/มิ.ย./2563   & DEV : สร้างส่วนของ Home Decor และแก้ส่วน Dni ใหม่ใช้ด้วยกันโดยแยก channel H, D  \\
	\hline
\end{longtable}

\begin{longtable}{|l|l|}
	\caption{บันทึกรายงานปฏิบัติงานประจำเดือน กรกฎาคม}\label{timeSheetDetailJulyJune} \\
	\hline
	\textbf{วันที่} & \textbf{รายละเอียดการทำงาน}            \\                                         
	\hline
	\endfirsthead
	\caption* {\textbf{ตารางที่ \ref{timeSheetDetailJulyJune} (ต่อ)} บันทึกรายงานปฏิบัติงานประจำเดือน กรกฎาคม} \\
	\hline
	\textbf{วันที่} & \textbf{รายละเอียดการทำงาน}            \\                                         
	\hline
	\endhead
	\hline
	\endfoot
	\hline
	01/ก.ค./2563    & DEV : สร้างส่วนของ Home Decor และแก้ส่วน Dni ใหม่ใช้ด้วยกันโดยแยก channel H, D                                                                           \\ 
	\hline
	02/ก.ค./2563    & DEV : สร้างส่วนของ Home Decor และแก้ส่วน Dni ใหม่ใช้ด้วยกันโดยแยก channel H, D                                                                           \\ 
	\hline
	03/ก.ค./2563    & ลาไปมหาวิทยาลัยเพื่อรับทุน                                                                                                                               \\ 
	\hline
	06/ก.ค./2563    & DEV : สร้างส่วนของ Home Decor และแก้ส่วน Dni ใหม่ใช้ด้วยกันโดยแยก channel H, D                                                                           \\ 
	\hline
	07/ก.ค./2563    & \begin{tabular}[c]{@{}l@{}} DS : group DivMng กับ Shippiont ที่เหมือนกัน \\~และเอาที่ status เป็น Y เมื่อค้นหามาใส่ระบบ fleet ว่าง\end{tabular}          \\ 
	\hline
	08/ก.ค./2563    & \begin{tabular}[c]{@{}l@{}} DS : group DivMng กับ Shippiont ที่เหมือนกัน \\~และเอาที่ status เป็น Y เมื่อค้นหามาใส่ระบบ fleet ว่าง\end{tabular}          \\ 
	\hline
	09/ก.ค./2563    & DS : ให้ตอนกำลัง Confirm Qfail กับ ปิดงาน เช็คว่า Q ถูกคอนเฟิร์มไปก่อนรึยัง                                                                              \\ 
	\hline
	10/ก.ค./2563    & DS : ให้ตอนกำลัง Confirm Qfail กับ ปิดงาน เช็คว่า Q ถูกคอนเฟิร์มไปก่อนรึยัง                                                                              \\ 
	\hline
	13/ก.ค./2563    & DS : เพิ่มอีเมลล์ในการแสดงผลหน้า ประวัติคิว และ หน้ารายละเอียดคิว                                                                                        \\ 
	\hline
	14/ก.ค./2563    & DS : เพิ่มอีเมลล์ในการแสดงผลหน้า ประวัติคิว และ หน้ารายละเอียดคิว                                                                                        \\ 
	\hline
	15/ก.ค./2563    & DS : แก้ fleet ที่หายไปจากหน้าจัดการทีมที่เปลี่ยนคันรถ                                                                                                   \\ 
	\hline
	16/ก.ค./2563    & DS : ให้เข้าไปเช็คว่า art qty กับ qty ตรงกันไหมยืนยันนำจ่ายหรือยัง ใน หน้าปิดงาน                                                                         \\ 
	\hline
	17/ก.ค./2563    & DS : ให้เข้าไปเช็คว่า art qty กับ qty ตรงกันไหมยืนยันนำจ่ายหรือยัง ใน หน้าปิดงาน                                                                         \\ 
	\hline
	20/ก.ค./2563    & DS : ให้เข้าไปเช็คว่า art qty กับ qty ตรงกันไหมยืนยันนำจ่ายหรือยัง ใน หน้าปิดงาน                                                                         \\ 
	\hline
	21/ก.ค./2563    & \begin{tabular}[c]{@{}l@{}} DS : แก้ javascript ให้ complete dialog \\ไม่สามรถกดปิดได้กดได้แค่ตกลง และทำให้ mouse scroll ไม่ได้ หน้าปิดงาน\end{tabular}  \\ 
	\hline
	22/ก.ค./2563    & \begin{tabular}[c]{@{}l@{}} SS : แก้ไขข้อมูล total non vat ใน pdf Department Sales Analysis Report\\~และ แก้ Excel รูปแบบข้อมูลบัตรเครดิต\end{tabular}   \\ 
	\hline
	23/ก.ค./2563    & \begin{tabular}[c]{@{}l@{}} SS : แก้ไขข้อมูล total non vat ใน pdf Department Sales Analysis Report \\~และ แก้ Excel รูปแบบข้อมูลบัตรเครดิต\end{tabular}  \\ 
	\hline
	24/ก.ค./2563    & SS : แก้เลขบัตรเครดิตให้เป็น XXXXXX ตรงกลาง 6 ตัวและนำตัวเลขมาใส่                                                                                        \\ 
	\hline
	29/ก.ค./2563    & SS : แก้เลขบัตรเครดิตให้เป็น XXXXXX ตรงกลาง 6 ตัวและนำตัวเลขมาใส่                                                                                        \\ 
	\hline
	30/ก.ค./2563    & SS : ไล่ดูว่ามันเปลี่ยนไปลง tax\_invoice ตอนไหนในตอนที่กด refund                                                                                         \\ 
	\hline
	31/ก.ค./2563    & SS : ไล่ดูว่ามันเปลี่ยนไปลง tax\_invoice ตอนไหนในตอนที่กด refund                                                                                         \\
	\hline
\end{longtable}

\begin{longtable}{|l|l|}
	\caption{บันทึกรายงานปฏิบัติงานประจำเดือน สิงหาคม}\label{timeSheetDetailJulyJuneAug} \\
	\hline
	\textbf{วันที่} & \textbf{รายละเอียดการทำงาน}            \\                                         
	\hline
	\endfirsthead
	\caption* {\textbf{ตารางที่ \ref{timeSheetDetailJulyJuneAug} (ต่อ)} บันทึกรายงานปฏิบัติงานประจำเดือน สิงหาคม} \\
	\hline
	\textbf{วันที่} & \textbf{รายละเอียดการทำงาน}            \\                                         
	\hline
	\endhead
	\hline
	\endfoot
	\hline
	03/ส.ค./2563    & ลากลับสถาบัน                                                                                                                                              \\ 
	\hline
	04/ส.ค./2563    & ลากลับสถาบัน                                                                                                                                              \\ 
	\hline
	05/ส.ค./2563    & ลากลับสถาบัน                                                                                                                                              \\ 
	\hline
	06/ส.ค./2563    & ลากลับสถาบัน                                                                                                                                              \\ 
	\hline
	07/ส.ค./2563    & ลากลับสถาบัน                                                                                                                                              \\ 
	\hline
	10/ส.ค./2563    & ลากลับสถาบัน                                                                                                                                              \\ 
	\hline
	11/ส.ค./2563    & \begin{tabular}[c]{@{}l@{}} SR : ปรับเงื่อนไขการดึงรายงานภาษีอย่างย่อ กรณีดึงข้อมูลที่ \\Center ปรับให้ดึงได้ 5 ปี\end{tabular}                           \\ 
	\hline
	13/ส.ค./2563    & ลากลับสถาบัน                                                                                                                                              \\ 
	\hline
	14/ส.ค./2563    & ลากลับสถาบัน                                                                                                                                              \\ 
	\hline
	17/ส.ค./2563    & ลากลับสถาบัน                                                                                                                                              \\ 
	\hline
	18/ส.ค./2563    & MEETING : กลับมาวันแรก คุยกับ HR                                                                                                                          \\ 
	\hline
	19/ส.ค./2563    & STUDY : ศึกษาเกี่ยวกับ Automate Test                                                                                                                      \\ 
	\hline
	20/ส.ค./2563    & MEETING : คุยปรึกษากับอาจารย์ถึงขนาดของโปรเจค                                                                                                             \\ 
	\hline
	21/ส.ค./2563    & STUDY : ศึกษาข้อมูลเกี่ยวกับ Device Farm                                                                                                                  \\ 
	\hline
	24/ส.ค./2563    & \begin{tabular}[c]{@{}l@{}} STUDY : ศึกษา Appium ซึ่งเป็น Driver Automate Test Mobile \\ที่ใช้กับ Device Farm\end{tabular}                                \\ 
	\hline
	25/ส.ค./2563    & \begin{tabular}[c]{@{}l@{}} STUDY : เขียน timeline ของการทำงานว่าจะแบ่งการทำงานเป็นอะไรยังไง\\และผลลัพธ์ได้อย่างไร\end{tabular}                           \\ 
	\hline
	26/ส.ค./2563    & \begin{tabular}[c]{@{}l@{}} STUDY : ศึกษา JavaScript(NodeJs) เบื้องต้นเพราะ Appium สามารถใช้ \\Nodejs ในการเขียน Script\end{tabular}                      \\ 
	\hline
	27/ส.ค./2563    & \begin{tabular}[c]{@{}l@{}} STUDY : ศึกษาหาข้อมูลเกี่ยวกับ Driver ที่เป็น library ที่ใช้ในการช่วย \\Test Flutter ใน Appium\end{tabular}                   \\ 
	\hline
	28/ส.ค./2563    & \begin{tabular}[c]{@{}l@{}} STUDY : ลองสร้าง Mobile App ที่เขียนขึ้นมาด้วยภาษา Dart(Flutter)\\~เพื่อที่จะนำไปทดลองใช้กับ Appium\end{tabular}              \\ 
	\hline
	31/ส.ค./2563    & \begin{tabular}[c]{@{}l@{}}STUDY : ทดลองเขียน TestScript ด้วย Library appium-flutter-driver \\แล้วนำไปทดสอบกับ Appium โดยทดสอบกับมือถือจริง\end{tabular}  \\
	\hline
\end{longtable}

\begin{longtable}{|l|l|}
	\caption{บันทึกรายงานการปฎิบัติงาน กันยายน}\label{timeSheetDetailJulyJuneAugSep} \\
	\hline
	\textbf{วันที่} & \textbf{รายละเอียดการทำงาน}            \\                                         
	\hline
	\endfirsthead
	\caption* {\textbf{ตารางที่ \ref{timeSheetDetailJulyJuneAugSep} (ต่อ)} บันทึกรายงานปฏิบัติงานประจำเดือน กันยายน} \\
	\hline
	\textbf{วันที่} & \textbf{รายละเอียดการทำงาน}            \\                                         
	\hline
	\endhead
	\hline
	\endfoot
	\hline
	01/ก.ย./2563    & \begin{tabular}[c]{@{}l@{}} STUDY : ศึกษาและทดลอง Mocha ซึ่งเป็น library ช่วยในการทำ \\TestScript ให้เขียนง่ายขึ้น\end{tabular}                                   \\ 
	\hline
	02/ก.ย./2563    & \begin{tabular}[c]{@{}l@{}} STUDY : ทดลองนำเอา Mocha และ TestScript ไปทดสอบใน \\Android Studio\end{tabular}                                                       \\ 
	\hline
	03/ก.ย./2563    & SIT2: Manual Test Web  Mobile                                                                                                                                     \\ 
	\hline
	08/ก.ย./2563    & SIT2: Manual Test Web  Mobile                                                                                                                                     \\ 
	\hline
	09/ก.ย./2563    & SIT2: Manual Test (Android)                                                                                                                                       \\ 
	\hline
	10/ก.ย./2563    & SIT2: Manual Test (Android)                                                                                                                                       \\ 
	\hline
	11/ก.ย./2563    & SIT2: Manual Test (Android)                                                                                                                                       \\ 
	\hline
	14/ก.ย./2563    & SIT2: Manual Test (Android)                                                                                                                                       \\ 
	\hline
	15/ก.ย./2563    & STUDY : ทดลองใช้ Device Farm กับ Appium                                                                                                                           \\ 
	\hline
	16/ก.ย./2563    & SIT2: Manual Test (Mobile) โทรศัพทร์ส่วนตัว                                                                                                                       \\ 
	\hline
	17/ก.ย./2563    & SIT2: Manual Test (Mobile) โทรศัพทร์ส่วนตัว                                                                                                                       \\ 
	\hline
	18/ก.ย./2563    & STUDY : ทดสอบและติดตั้ง ระบบ ลง window (ที่ผ่านมาใช้บน Ubuntu)                                                                                                    \\ 
	\hline
	21/ก.ย./2563    & STUDY : ศึกษาและทดลองหาวิธีในการทดสอบแอพโดยการไม่แก้ไขโค้ดเดิม (ทำไม่ได้)                                                                                         \\ 
	\hline
	22/ก.ย./2563    & STUDY : ลองแก้ไขโปรเจคและใส่ Key ให้ทำงานกับ Test Script ได้                                                                                                      \\ 
	\hline
	23/ก.ย./2563    & SIT2: เขียน Test Script ทดสอบหมวด Log-in                                                                                                                          \\ 
	\hline
	24/ก.ย./2563    & \begin{tabular}[c]{@{}l@{}} SIT2: Manual Testหาวิธีแก้การต้องกด Permission Allow ทุกครั้งเมื่อรัน App Test \\ให้สามารถ Allow ด้วยตัวเองได้ไม่ต้องกด\end{tabular}  \\ 
	\hline
	25/ก.ย./2563    & SIT2: Manual Testเขียน Test Script หน้าหลักในการ เช็ค Category ของ E-Catalog                                                                                      \\ 
	\hline
	28/ก.ย./2563    & SIT2: Manual Testเขียน Test Script หน้าหลักในการ เช็ค Category ของ E-Catalog                                                                                      \\ 
	\hline
	29/ก.ย./2563    & SIT2: จัดรูปแบบโค้ดใหม่และทำ Search Bar Test                                                                                                                      \\ 
	\hline
	30/ก.ย./2563    & SIT2: Test หมวดย่อย level2 (เครื่องใช้ไฟฟ้า)                                                                                                                      \\
	\hline
\end{longtable}

\begin{longtable}{|l|l|}
	\caption{บันทึกรายงานการปฎิบัติงาน ตุลาคม}\label{timeSheetDetailJulyJuneAugSepOct} \\
	\hline
	\textbf{วันที่} & \textbf{รายละเอียดการทำงาน}            \\                                         
	\hline
	\endfirsthead
	\caption* {\textbf{ตารางที่ \ref{timeSheetDetailJulyJuneAugSepOct} (ต่อ)} บันทึกรายงานปฏิบัติงานประจำเดือน ตุลาคม} \\
	\hline
	\textbf{วันที่} & \textbf{รายละเอียดการทำงาน}            \\                                         
	\hline
	\endhead
	\hline
	\endfoot
	\hline
	01/ต.ค./2563    & SIT2: เช็ค Category Level2 เพิ่มอีก 6 category                                                                                                      \\ 
	\hline
	02/ต.ค./2563    & SIT2: เช็ค Category Level2 ที่เหลือจนเสร็จ                                                                                                          \\ 
	\hline
	05/ต.ค./2563    & SIT2: เช็ค ปุ่มกด"ทั้งหมด"ใน Category ว่าถูกไหม                                                                                                     \\ 
	\hline
	06/ต.ค./2563    & SIT2: เทสการกรองหน้าทั้งหมวด (ยกเว้นกรองแบบ ช่วงราคา)                                                                                               \\ 
	\hline
	07/ต.ค./2563    & SS : แก้ปัญหา SQL ตอนกดดาวน์โหลด execel ใน MYCENTER                                                                                                 \\ 
	\hline
	08/ต.ค./2563    & \begin{tabular}[c]{@{}l@{}} SS : แก้ MY01 ตอนกรอกที่อยู่ให้ที่อยู่เลือกภาษาไทยหรืออังกฤษได้\\และแก้ HTML ที่กรอกที่อยู่\end{tabular}                \\ 
	\hline
	09/ต.ค./2563    & \begin{tabular}[c]{@{}l@{}} SS : แก้ MY01 ตอนกรอกที่อยู่ให้ที่อยู่เลือกภาษาไทยหรืออังกฤษได้\\และแก้ HTML ที่กรอกที่อยู่\end{tabular}                \\ 
	\hline
	12/ต.ค./2563    & SIT2 : ทำ Test Script ในหมดของรายละเอียดสินค้า                                                                                                      \\ 
	\hline
	14/ต.ค./2563    & ลาไปรับรถ                                                                                                                                           \\ 
	\hline
	15/ต.ค./2563    & DS : หาจุดที่ API ผิดพลาดตอนประเมิณช่าง                                                                                                             \\ 
	\hline
	16/ต.ค./2563    & DS : ในตอนจะยืนยันช่างปิดงานด้วยรูปไม่มีให้เลือกว่า ผ่านหรือไม่ผ่าน                                                                                 \\ 
	\hline
	19/ต.ค./2563    & SIT2 : ทำ Test Script ในหมดของรายละเอียดสินค้า                                                                                                      \\ 
	\hline
	20/ต.ค./2563    & SIT2 : ทำ Test Script ในหมดของรายละเอียดสินค้า                                                                                                      \\ 
	\hline
	21/ต.ค./2563    & DS : ในตอนจะยืนยันช่างปิดงานด้วยรูปไม่มีให้เลือกว่า ผ่านหรือไม่ผ่าน                                                                                 \\ 
	\hline
	22/ต.ค./2563    & \begin{tabular}[c]{@{}l@{}} DS : หาว่าเมื่อจัดทีมช่างและนำคิวย้ายช่างส่งเข้าไปถึงไหนในระบบ TMS \\และสถานะที่เปลี่ยนเป็น CC อยู่ที่ไหน\end{tabular}  \\ 
	\hline
	23/ต.ค./2563    & \begin{tabular}[c]{@{}l@{}} DS : หาว่าเมื่อจัดทีมช่างและนำคิวย้ายช่างส่งเข้าไปถึงไหนในระบบ TMS \\และสถานะที่เปลี่ยนเป็น CC อยู่ที่ไหน\end{tabular}  \\ 
	\hline
	26/ต.ค./2563    & SIT2 : ทำ Test Script เปลี่ยนวิธีมาใช้ Xpath ในหมวดของ เปรียบเทียบ                                                                                  \\ 
	\hline
	27/ต.ค./2563    & SIT2 : ทำ Test Script เปลี่ยนวิธีมาใช้ Xpath ในหมวดของ เปรียบเทียบ                                                                                  \\ 
	\hline
	28/ต.ค./2563    & \begin{tabular}[c]{@{}l@{}} DS : แก้ให้ comma สามารถลงคอลัมได้ตอนนำ csv ออกมา และ หาจุด \\error ในหน้า แก้ไขข้อมูลลูกค้า\end{tabular}               \\ 
	\hline
	29/ต.ค./2563    & DEV : เพิ่ม Report Sales แบบ API อยู่ในโปรเจค SalesInformation                                                                                      \\ 
	\hline
	30/ต.ค./2563    & DEV : เพิ่ม Report Sales แบบ API อยู่ในโปรเจค SalesInformation                                                                                      \\
	\hline
\end{longtable}
	
	

\clearpage 
\thispagestyle{empty}
\begin{center}
	\vspace*{\stretch{1}}
	\LARGE{\textbf{ภาคผนวก ค}}
	\vspace*{\stretch{1}}
\end{center}

\chapter{ประวัติผู้เขียน}
\begin{tabularx}{\linewidth}{lX}
	ชื่อ-นามสกุล&นาย เสฎฐวุฒิ ไม้สนธิ์\\
	วัน เดือน ปี เกิด&13 สิงหาคม พ.ศ.2541\\
	ที่อยู่& บ้านเลขที่ 138 ซอยสายลม ถนนพหลโยธิน 
	\\
	&แขวงสามเสนใน เขตพญาไท กรุงเทพมหานคร 10400
	\\
	ประวัติการศึกษา&วิทยาศาสตรบัณฑิต สาขาวิชาเทคโนโลยีสารสนเทศ
	\\
	&คณะเทคโนโลยีสารสนเทศ
	\\
	&สถาบันเทคโนโลยีพระจอมเกล้าเจ้าคุณทหารลาดกระบัง
	\\
	\\
\end{tabularx}

    \nocite{*}
    
\end{document}