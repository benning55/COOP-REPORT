\chapter{ปัญหาและข้อเสนอแนะ}
\thispagestyle{empty}
\label{chapter:result}

\section{ปัญหาด้านสถานประกอบการ}
    \subsection{ปัญหา}
        ปัญหาในการสื่อสารภายในองค์กรของฝ่ายบุคคลกับแผนกที่รับสหกิจศึกษาเริ่มจากการที่แผนกเคยชี้แจงไว้ว่าไม่ขอรับสหกิจศีกษาแต่ก็ยังส่งมาในแผนกที่ไม่ต้องการอยู่ดี
    \subsection{ข้อเสนอแนะ}
        ในอนาคตหากต้องรับนักศึกษาสหกิจศึกษาจึงอยากที่จะให้ทางฝ่ายบุคคลเคารพการตัดสินใจของแผนกที่ไม่ต้องการรับและหาแผนกที่สามารถรองรับได้หรือหากไม่มีขอให้ปฎิเสธนักศีกษาอย่างไวที่สุด

\section{ปัญหาด้านสถาบัน}
    \subsection{ปัญหา}
        ปัญหาการในการให้ความเข้าใจกับสถานประกอบการว่าโครงการสหกิจต้องเป็นงานที่มีปริมาณขนาดไหนทำให้เตรียมไม่ถูก
    \subsection{ข้อเสนอแนะ}
        อยากให้ทางสถาบันออกเป็นรายการตรวจสอบไว้เพื่อให้ทางสถานประกอบการได้ไว้เพื่อใช้ประกอบการตัดสินใจรับนักศึกษาเข้าโครงการสหกิจศีกษาว่ามีโครงงานตามที่โครงการสหกิจต้องการหรือไม่
\section{ปัญหาด้านตัวนักศึกษา}
    \subsection{ปัญหา}
        ปัญหาของนักศึกษาคือการสื่อสารกับพี่ที่เป็นที่ปรึกษาเนื่องจากไม่ทราบจะเข้าหายังไงให้ถูกจังหวะ
    \subsection{ข้อเสนอแนะ}
        นักศึกษาควรสังเกตการทำงานของพี่ที่จะเข้าไปสื่อสารด้วยให้มากกว่านี้เพื่อหาเวลาในการเข้าไปสื่อสารอย่างไม่ผิดกาละเทศะ
