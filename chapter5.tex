\chapter{บทสรุป}
\thispagestyle{empty}
\label{chapter:conclusion}

% \section{วิเคราะห์ผลการปฏิบัติงาน}
% จากการที่นักศึกษาปฏิบัติงานสหกิจศึกษา ณ ธนาคาร ซีไอเอ็มบีไทย จำกัด (มหาชน) ตั้งแต่วันที่ \StartDWork จนถึง \EndDWork 
% รวมเป็นเวลา 5 เดือน 17 วัน ซึี่งนักศึกษาได้มีส่วนร่วมในการช่วยพัฒนาโมบายล์แอปพลิเคชันบนระบบปฏิบัติการ Android และ iOS
% ในส่วนของการจองซื้อหุ้นกู้นั้น ทำให้ข้าพเจ้าได้รับประสบการณ์หลากหลายรูปแบบที่สอนให้ข้าพเจ้าได้รู้จักการทำงานแบบทีม ไม่ว่าจะเป็นการ
% คุยงาน วางแผน กำหนดขอบเขตการพัฒนา ข้อผิดพลาดต่าง ๆ ที่เกิดขึ้นระหว่างพัฒนา และการแก้ปัญหาที่เกิดขึ้นจริงระหว่างพัฒนา 
% อีกทั้งยังได้เรียนรู้เทคโนโลยีต่าง ๆ ที่ช่วยให้เราพัฒนาซอฟต์แวร์ได้มีประสิทธิภาพมากขึ้น นอกจากนี้ยังได้ทักษะการสื่อสารกับคนในทีมทั้งรูปแบบการพูด
% และเรียนรู้การใช้ภาษาอังกฤษในการทำงาน สุดท้ายการที่ได้มาร่วมงาน ณ ที่แห่งนี้ทำให้ข้าพเจ้า ได้เพิ่มทักษะในการวิเคราะห์ เพิ่มความมีระเบียบวินัย
% รู้จักความรับผิดชอบในหน้าที่ของตนเองมากขึ้น ซึ่งทั้งหมดที่กล่าวมานั้นจะหล่อหลอมให้ข้าพเจ้าสามารถนำประสบการณ์ และความรู้นี้ไปใช้ในการทำงานจริงได้ในอนาคต

% \section{ประโยชน์ที่ได้รับจากการปฏิบัติงาน}
% \subsection{ประโยชน์ต่อตนเอง} 
% พัฒนาความรู้ความสามารถจากการนำเรื่องที่ศึกษาในคณะเทคโนโลยีสารสนเทศมาต่อยอดในสถานประกอบการจริง
% อีกทั้งยังสามารถนำเรื่องต่าง ๆ ที่เรียนรู้มาใช้ต่อในอนาคตได้อีกด้วย

% \subsection{ประโยชน์ต่อสถานประกอบการ}
% สถานประกอบการมีช่องทางการจองซื้อหุ้นกู้เพิ่มจากเดิมที่ต้องทำการจองซื้อผ่านสาขาอย่างเดียว ทำให้ลูกค้ามีความสะดวกและพึงพอใจมากขึ้น
% อีกทั้งยังเป็นการเพิ่มรายได้ให้แก่สถานประกอบการอีกด้วย

% \subsection{ประโยชนต่อมหาวิทยาลัย}
% ทำให้สถานประกอบการมองว่ามีนักศึกษาที่สามารถทำงานได้จริงในปัจจุบันก่อนศึกษาจบ 
% เพิ่มความเชื่อให้แก่สถาณประกอบการต่าง ๆ ว่านักศึกษาที่มาจากสถาบันนี้มีคุณภาพ

% \section{วิเคราะห์จุดเด่น จุดด้อย โอกาส อปสรรค (SWOT Analysis)}
% \subsection{จุดเด่น}  
% เป็นคนที่ใช้เวลาในการเรียนรู้น้อย และเข้าใจสิ่งต่าง ๆ ได้ง่าย สามารถเริ่มงานได้ไว ตั้งใจทำงานอย่างเต็มที่
% มีความมุ่งมั่นสูง มีความคิดในการแก้ปัญหาเมื่อทีมเจอปัญหา

% \subsection{จุดด้อย}
% เป็นคนที่ติดการถามก่อนทำจริงซึ่งในโลกแห่งความเป็นจริงแล้วไม่มีใครสอนเราได้ดีกว่าการเรียนรู้ด้วยตนเอง
% เป็นคนที่มีความคิดที่ไม่ถี่ถ้วน คิดไม่รอบคอบ ชอบทำอะไรเกินความต้องการเมื่อมีคำสั่งมา
% ปัญหา

% \subsection{โอกาส}
% ได้มีเรียนรู้สิ่งใหม่ ๆ ทั้งเรื่องของความรู้ สังคม และการใช้ชีวิตในการทำงานจริง
% ความรู้ได้ทั้งการเรียนรู้การพัฒนา Mobile Application ที่ตัวนักศึกษานั้นไม่เคยได้ทำมาก่อน
% ทั้ง iOS และ Android นักศึกษาได้มีโอกาสสื่อสารภาษาอังกฤษบ่อยครั้งเนื่องจากมีเพื่อนร่วมที่เป็นคนต่างชาติ

% \subsection{อุปสรรค}
% นักพัฒนาในทีมน้อยเกินไปทำงานหนักไปในบาง Sprint ที่มี Issue เยอะ ๆ ทำให้ต้องทำงานร่วงเวลาในบางครั้ง
% และยังมีการที่มี Dependencies กับทีมอื่นที่ Block การพัฒนาของทีมจึงทำให้เกิดการพัฒนาที่ช้าในบางส่วน

% \section{ปัญหา และข้อเสนอแนะ}
% \subsection{ด้านมหาวิทยาลัย}
% ระบบจัดการนักศึกษาสหกิจศึกษาไม่สมบูรณ์ขาดความชัดเจนในหลาย ๆ ส่วนทำให้นักศึกษา และสถานประกอบการไม่เข้าใจว่า
% ใครจะเป็นคนประสานงาน อีกทั้งยังมีการประชาสัมพันธ์ที่ไม่ชัดเจน เอกสารต่าง ๆ ที่ไม่มีมาตรฐานที่ชัดเจน 

% \noindent \textbf{ข้อเสนอแนะ:} ควรปรับปรุงระบบนักศึกษาสหกิจศึกษาให้ดีกว่านี้ และอยากให้เน้นการปฏิบัติงาน
% ของนักศึกษามากกว่าการดูในผลงานที่เป็นชิ้นงาน เนื่องจากอาจจะไม่มีชิ้นงานที่เป็นกิจลักษณะ 
% อยากให้ดูว่านักศึกษาได้อะไรจากฝึกงานมากกว่า

% \subsection{ด้านตัวนักศึกษา}
% เป็นคนถามก่อนที่จะได้ลงมือปฏิบัติงานจริง เป็นคนตอบตกลงโดยไม่ประเมินก่อนว่าตนเองนั้นทำได้หรือทำไม่ได้
% เป็นคนพูดเร็ว คิดแล้วทำอะไรไวเกินไป ทำให้การสื่อสารบางส่วนกับคนในทีมไม่เข้าใจ หรือไม่ตรงกัน

% \noindent \textbf{ข้อเสนอแนะ:} ควรปรับปรุงในเรื่องของความการลองลงมือทำก่อนมีปัญหาแล้วลองแก้ไขเอง 
% ถ้าแก้ปัญหาแล้วทำไมได้ค่อยถาม ต้องประเมินงานก่อนที่จะตกลงในทุก ๆ ด้านว่าตนเองทำได้หรือไม่ และต้องเป็นคนที่พูดช้ากว่านี้
% เพื่อที่การสื่อสารจะได้ดีกว่านี้

% \subsection{ด้านสถานประกอบการ}
% เนื่องจากสถานประกอบการได้รับนักศึกษาสหกิจศึกษาเป็นครั้งแรก (ในแผนก Digital Banking) ทำให้ยังไม่ค่อยรู้เรื่องวิธีการ
% จัดการกับนักศึกษาและโปรเจคทำให้เรื่องการหาโปรเจคให้นักศึกษานั้นล่าช้า และ Scope ของโปรเจคยังไม่ชัดเจน และมีจำนวน

% \noindent \textbf{ข้อเสนอแนะ:} ควรมีโปรเจคที่ Scope ชัดเจนกว่านี้ที่ให้นักศึกษาได้ทำ
